\begin{thebibliography}{99}
\bibitem{ref_1_1_GS} Shaoqiang Chen, Masahiro Yoshita, Takashi Ito, Toshimitsu Mochizuki, Hidefumi Akiyama, Hiroyuki Yokoyama, Kenji Kamide, and Tetsuo Ogawa. Analysis of Gain-Switching Characteristics Including Strong Gain Saturation Effects in Low-Dimensional Semiconductor Lasers. \sl Japanese Journal of Applied  Physics \rm , Vol 5,098001, 2012.

\bibitem{ref_iga} 伊賀健一 (1994)『半導体レーザ』, オーム社
\bibitem{ref_t_ito}Takashi Ito
, Hidekazu Nakamae, Yuji Hazama, Takahiro Nakamura, Shaoqiang Chen, Masahiro Yoshita, Changsu Kim, Yohei Kobayashi and Hidefumi Akiyama.
Femtosecond pulse generation beyond photon lifetime limit in gain-switched semiconductor lasers .\sl Communications Physics\rm , DOI: 10.1038/s42005-018-0045-0, 2018. 

\bibitem{ref_hasegawa} A. Hasegawa and Y.Kodama, Signal transmission by optical solitons in monomode fiber, \sl Proc. IEEE\rm ,vol.69, No.9, 1145-1150, (1981)
\bibitem{ref_chichkov} B.N. Chichkov, C.momma, S.Nolte,F. von Alvensleben, A. T$\rm{\ddot{u}}$nnermann, Femtosecond, picosecond and nanosecond laser ablation of solids. \sl Appl. Phys. A\rm vol.63,109-115, 1996.
\bibitem{ref_h_ito} Hiromasa Ito, Shigeru 
murata, Hiroyuki Yokoyama and Humio Inaba. 高周波変調にもとづくAlGaAs半導体レーザーからの超短光パルス発生, 応用物理, 第50巻, 第1号, 18-28, (1981)
\bibitem {ref_lau} K. Y. Lau. Gain awitching of semiconductor Injection lasers, \sl Alli. Phys. Lett. \rm , vol.53, 257,1987.
\bibitem{ref_t_kawamura}河村力(1996)『セラミックス基礎講座6 結晶と電子』、内田老鶴圃
\bibitem{ref_n_mikoshiba}御子柴宣夫(1982)『半導体工学シリーズ2 半導体の物理』、培風館
\bibitem{ref_Matthews} J.W.Matthews : J.Crystal Growth, vol.27,118,(1974)
\bibitem{ref_konagai}小長井誠,『半導体超格子入門』1987,培風館
\bibitem{ref_thijs} P.J.Thijs, J.J.M.Binsma, L.F. Tiemejier, and T. van Dorgen ,\sl Electron . Lett.\rm vol 28, issue 9, 829,(1992)
\bibitem{ref_Dutta}N. K. Dutta, Fellow, W. S. Hobson, D. Vakhshoori, H. Ham, P. N. Freeman, J. F. de Jong, and J, Lopata \sl IEEE PHOTONICS TECHNOLOGY LETTERS\rm ,vol. 8, No. 7, (1996)
\end{thebibliography}
