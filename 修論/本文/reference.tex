\begin{thebibliography}{99}
\bibitem{ref_1_1_GS} Shaoqiang Chen, Masahiro Yoshita, Takashi Ito, Toshimitsu Mochizuki, Hidefumi Akiyama, Hiroyuki Yokoyama, Kenji Kamide, and Tetsuo Ogawa. Analysis of Gain-Switching Characteristics Including Strong Gain Saturation Effects in Low-Dimensional Semiconductor Lasers. \sl Japanese Journal of Applied  Physics \rm , Vol 5,098001, 2012.

\bibitem{ref_iga} 伊賀健一 (1994)『半導体レーザ』, オーム社
\bibitem{ref_t_ito}Takashi Ito
, Hidekazu Nakamae, Yuji Hazama, Takahiro Nakamura, Shaoqiang Chen, Masahiro Yoshita, Changsu Kim, Yohei Kobayashi and Hidefumi Akiyama.
Femtosecond pulse generation beyond photon lifetime limit in gain-switched semiconductor lasers .\sl Communications Physics\rm , DOI: 10.1038/s42005-018-0045-0, 2018. 

\bibitem{ref_hasegawa} A. Hasegawa and Y.Kodama, Signal transmission by optical solitons in monomode fiber, \sl Proc. IEEE\rm ,vol.69, No.9, 1145-1150, (1981)
\bibitem{ref_chichkov} B.N. Chichkov, C.momma, S.Nolte,F. von Alvensleben, A. T$\rm{\ddot{u}}$nnermann, Femtosecond, picosecond and nanosecond laser ablation of solids. \sl Appl. Phys. A\rm vol.63,109-115, 1996.
\bibitem{ref_h_ito} Hiromasa Ito, Shigeru 
murata, Hiroyuki Yokoyama and Humio Inaba. 高周波変調にもとづくAlGaAs半導体レーザーからの超短光パルス発生, 応用物理, 第50巻, 第1号, 18-28, (1981)
\bibitem {ref_lau} K. Y. Lau. Gain awitching of semiconductor Injection lasers, \sl Alli. Phys. Lett. \rm , vol.53, 257,1987.
\bibitem{ref_t_kawamura}河村力(1996)『セラミックス基礎講座6 結晶と電子』、内田老鶴圃
\bibitem{ref_n_mikoshiba}御子柴宣夫(1982)『半導体工学シリーズ2 半導体の物理』、培風館
\bibitem{ref_Matthews} J.W.Matthews : J.Crystal Growth, vol.27,118,(1974)
\bibitem{ref_konagai}小長井誠,『半導体超格子入門』1987,培風館
\bibitem{ref_thijs} P.J.Thijs, J.J.M.Binsma, L.F. Tiemejier, and T. van Dorgen ,\sl Electron . Lett.\rm vol 28, issue 9, 829,(1992)
\bibitem{ref_Dutta}N. K. Dutta, Fellow, W. S. Hobson, D. Vakhshoori, H. Ham, P. N. Freeman, J. F. de Jong, and J, Lopata \sl IEEE PHOTONICS TECHNOLOGY LETTERS\rm ,vol. 8, No. 7, (1996)
\bibitem{ref_h_yokoyama} Hiroyuki Yokoyama, Hengchang Guo , Takuya Yoda, Keijiro Takashima, Ki-Ichi SaTo, Hirokazu, Taniguchi, anc Hiromasa Ito, Two-photon bioimaging with picosecond optical pulses from a semiconductor laser,\sl OPTICS EXPRESS\rm , vol. 14, No.8,2006
\bibitem{ref_aspin} G.J. Aspin ,J.E.Carroll,and R. G. Plumb . The effect of cavity lenght on picosecond pulse generation with highly rf modulated AlGaAs double heterostructure lasers. \sl Appl. Phys. Lett\rm , Vol. 39, p. 860, 1981.
\bibitem{ref_torphammar} P. Torphammar, S. T. ENG. PICOSECOND PULSE GENERATION IN SEMICONDUCTOR LASERS USING RESONANCE OSCILLATION. \sl ELECTRONICS LETTERS\rm , vol. 16, No. 15, p. 587, 1980 
\bibitem{ref_dupuis} R. D. Dupuis, P. D. Dapkus, N. Holonyak and R.M.Kolbas. Continuous room-temperture multiple-quantum-well $\rm{Al_{x}Ga_{1-x}As-GaAs}$ injection lasers grown by metalorganic chemical vapor deposition, \sl  Appl. Phys. Lett.\rm , Vol. 35, p.487, 1979.
\bibitem{ref_van} J. P. van der Ziel, Dingle, R. C. Miller, W. Wiegman, and W. A. Nordland. Laser oscillation from quantum state in very thin $\rm{GaAs-Al_{0.2}Ga_{0.8}As }$multilayer structures. \sl Appl. Phys. Lett.\rm , Vol. 26,  p. 463, 1975. 
\bibitem{y_arakawa} Yasuo Arakawa, and Amnon Yariv. Theory of Gain, Mdulation Response, and Spectral Linsewidth in AlGaAs Quantum Well Lasers. \sl IEEE JOURNAL OF QUANTUM ELECTRONICS\rm , Vol. QE-21, No. 10, 1985.
\bibitem{ref_band_para} I. Vurgaftman, J. R. Meyer, and L. R. Ram-Mohan. Band parameters for III-V compounds semiconductors and their alloys. \sl Journal of Applied Physics\rm , Vol. 89, p. 5815, 2001.
\end{thebibliography}
