\documentclass[11pt]{jreport}
\usepackage{amsmath,amssymb,amsfonts}
\usepackage{bm}
\usepackage{siunitx}
\usepackage[dvipdfmx]{graphicx}
\usepackage[top=30truemm,bottom=30truemm,left=25truemm,right=25truemm]{geometry}
\usepackage{comment}


\title{
{\LARGE 修士論文\\[1cm]}
{\huge InGaAs系高利得量子井戸半導体レーザーの開発\\及び評価測定\\[1cm]
Development and characterization measurements of \\InGaAs high-gain quantum-well
lasers\\[5cm]}
}
\date{\LARGE 平成31年1月4日}
\author{\LARGE 東京大学大学院理学系研究科\\
\LARGE 物理学専攻\\[1cm]
\LARGE 指導教員 秋山 英文 教授\\[1cm]
\LARGE 学籍番号 35-176043\\
\LARGE 小松原 望}
\begin{document}
\maketitle			%タイトル
\begin{abstract}
 半導体レーザーは世の中で広く用いられている発光デバイスである。小型、高安定などの利点を有し非常に扱いが容易である。そのような半導体デバイスから直接ps程度の超短パルスを発生させることは、超短パルスを用いた様々な基礎研究の成果を技術的に応用する足がかりとなり得る。
 
 半導体から超短パルス光を得る手法として利得スイッチングを取り上げる。
 利得スイッチングはns程度の短い電気パルスを注入することそれより短い数十psの光パルスを得る技術である。電流による直接変調のみで実現可能でありまた複雑な構造を必要としないことから簡便に実現が可能な手法といえる。
 %電流注入により半導体内部の利得を増幅し、反転分布を作る。
 
 利得スイッチングパルスによるパルス発生に関しては古くから研究がなされてきているが、近年の研究で光パルスの立ち上がりは半導体材料の利得飽和が、立ち下がりは共振器寿命が決めているということが報告されている。共振器寿命は半導体レーザーの共振器長を短くすることで短パルス化が見込まれる。一方利得飽和に関してはレーザーデバイス設計の時点では決めることができない。そこで利得飽和の代わりに光閉じ込め係数を高くすることで高い利得を実現しパルスの短パルス化を図れることが示唆されている。量子井戸レーザーにおいて閉じ込め係数は量子井戸の数を増やすことで実現される。
 

 そこで本研究では10周期多重量子井戸半導体レーザーをデザインおよび開発を行った。同時に3周期多重量子井戸レーザーも開発し、電流注入実験を行い比較したところ高い利得であることが示された。
 
 また利得スイッチング動作を試みたところ10周期多重量子井戸レーザーでは最短で26.5 psというパルス幅を与えた。
 
 %共振器長あるいは量子井戸の層数を変えたことによる短パルス化の効果はあまり見られなかった。
\end{abstract}	%要旨
\tableofcontents 	%目次
\listoffigures 		%図目次
\listoftables 		%表目次

%%%%%%%%%%%%%%%%%%%%%%%%%%%%%%%%%%%%%%%%%%%%%%%%
%%%  本文
%%%%%%%%%%%%%%%%%%%%%%%%%%%%%%%%%%%%%%%%%%%%%%%

% !TEX root = main.tex
\chapter{序論}
\section{研究背景}
\subsection{半導体レーザー}
\subsubsection{半導体レーザーの現状}
半導体レーザーが実現されたのは1962年のことである 。キャリアと光の閉じ込めを能率よくできるようにした2重ヘテロ構造が用いられ流ようになり実用化・発展を遂げた。光通信、光ディスク用発光デバイスの核をなす技術である。他のレーザーにと比較しても小型・軽量、大量生産可能、熱や振動(安定性)に強い、高い発振波長選択性などが主な理由である。近年では半導体からピコ秒程度の超短光パルスを発生させる技術も研究が盛んに行われており、産業への応用が期待されている。
%[利得スイッチングsds生物発光yokoyamaさん,ここではGSにこだわらなくてもいいか][psパルスを使った産業]
\subsubsection{超短パルス発生}
ピコ秒オーダーの超短パルスを発生する技術は長距離光ファーバー伝送\cite{ref_hasegawa}に加えて、精密レーザー加工\cite{ref_chichkov} や多光子励起顕微鏡を用いたバイオイメージング[]など、応用の幅が広がってきている技術である。

半導体レーザーを用いた短パルス発生の代表的な方法としては利得スイッチングとモード同期法がある。利得スイッチング\cite{ref_h_ito}は注入電流を変調する直接変調を用いた方法である。デバイスにナノ秒程度の電流パルスを注入すると励起パルスよりも短い、数十psの光パルスが得られるというものである。半導体内の光強度が大きくなると誘導放出によって利得が急激に減少するためである。特徴としては複雑な構造を必要とせずずべての半導体レーザーで実現可能な技術であるという点である。

一方のモード同期法はサブps程度の超短パルスを得ることができる技術である。外部共振器あるいは共振器内に過飽和吸収体を挿入するなど付加的な構造が必要となる。

本研究では比較的容易に実現できる利得スイッチングに注目する。
\begin{comment}
 半導体から短パルスを発生させる方法として従来行われてきた方法が主に3つある。モード同期法、Qスイッチ法、利得スイッチ法である。モード同期法は個体レーザーでも用いられておりフーリエ限界に近いパルス幅を生成できる反面パルスの繰り返し周期が固定されてしまうという特徴を持つ。一方Qスイッチング法は過飽和吸収帯を用いるなどしてQ値を瞬間的に増大させることで高エネルギーの光パルスを得ることができる。利得スイッチングは電流を変調する直接変調の一種であり、複雑な構造を必要とせず、全ての半導体レーザーで実現が可能な技術である。レーザー加工などの技術的応用においては繰り返し可変であることや様々な種類の光源を試すことができるという利点があるため、??本研究では利得スイッチング法に着目した。
\end{comment}


\subsubsection{利得スイッチング法}
利得スイッチングは光通信技術を念頭に置いた短パルス光発生の進歩の過程で観測された。その後電源として用いるパルスジェネレーターの進歩などにより研究が進み、報告も盛んに行われてきた。図\ref{fig:fig_1_1_GS_ito}bに過去40年間に報告されてきた利得スイッチングのパルス幅がプロットされている。赤が光励起、青が電流注入を表している。電流注入の報告を見てみると最短でも5ps程度である。

また、応用としては利得スイッチングを利用した生物発光イメージの撮像の報告がある[]


次に利得スイッチングの基本的な動作について述べる。


利得スイッチング動作は半導体中のレーザー動作特性を記述するレート方程式により理解されている\cite{ref_lau}。
レート方程式はデバイス中のキャリア密度と光子密度の時間変化の振る舞いを以下のような連立方程式で表している。nは量子井戸1層あたりのキャリア密度、sは全活性層の光子密度を表す。式(\ref{eq:late_eq_1})はnの時間変化を記述している。右辺第1項は外部から注入されるポンプキャリア、第2項は誘導放出、第3項は自然放出を表す。式(\ref{eq:late_eq_2})は光子の時間変化を記述しており、第1項は誘導放出による増幅、第2項は光子寿命による減衰(共振器寿命)、第3項は自然放出光による増幅を表す。
\begin{eqnarray}
\dfrac{dn}{dt}&=&n_{\rm{pump}}\zeta(t)-\dfrac{\Gamma}{m}\nu_{\rm{g}}g(n)\dfrac{s}{1+\epsilon s}-\dfrac{n}{\tau_{r}}\\
\label{eq:late_eq_1}
\dfrac{ds}{dt}&=&\Gamma\nu_{g}g(n)\dfrac{s}{1+\epsilon s}-\dfrac{s}{\tau_{p}}+m\beta\dfrac{n}{\tau_{r}}
\label{eq:late_eq_2}
\end{eqnarray}
\begin{eqnarray*}
&n& : 量子井戸1層あたりのキャリア密度 [m^{-3}]\\
&s& : 活性層全体の光子密度[m^{-3}]\\
&n_{\rm{pump}}& : 励起キャリア密度 \\
&\zeta(t)& : 規格化された励起パルスの時間変化\\
&\Gamma & : 光閉じ込め係数\\
&m& : 量子井戸数\\
&\nu_{g}& : 群速度[m/s]=c/n_{eq}?\\
&g(n)& : 利得[cm^{-1}]\\
&\epsilon & : 利得圧縮係数?\\
&\tau_{r}& : キャリア寿命[s^{-1}]\\
&\tau_{p}& : 光子寿命 [s^{-1}]\\
&\beta& : 自然放出光係数\\
\end{eqnarray*}


上式のようなレート方程式を基に短い励起パルスを印可した時の発光および利得の時間変化についてシミュレーションを行った結果を図\ref{fig:fig_1_1_GS_ito}aに示す。赤線が光励起によるインパルス励起の様子、青線が電流注入による数ns秒パルス励起の様子である。青線に注目すると図\ref{fig:fig_1_1_GS_ito}a上段での励起パルスよりも短い、数十ps光パルスが出てくることがわかる。さらに1つ目のパルスの後は緩和振動が起きている。これが典型的な利得スイッチング動作である。また下段には利得の時間変化が示されている。励起が始まると同時に電子密度つまり利得が増えていき、ある時刻をすぎると閾値に達し反転分布を形成する。今度は誘導放出によって一気にキャリアが放出される。このキャリアの消費が注入されるキャリアよりも大きくなるため、利得も急激に減衰し、光パルスも急峻に立ち下がる。これが利得スイッチングの理解である。

\begin{figure}[h]
	\centering
	\includegraphics[width=15cm]{figure/fig_1_1_GS_ito.png}
	\caption{a 利得スイッチングのメカニズム, b過去の研究におけるパルス幅\cite{ref_t_ito}}
	\label{fig:fig_1_1_GS_ito}
\end{figure}

レート方程式の誘導放出の項に関わってくる利得$g(n)$はキャリア密度nに比例する線形利得$g_{0}(n-n_{0})$のような形で近似されてきた($g_{0}$は利得定数,$g_{0}$は透明キャリア密度)。しかしchenらはg{n}に非線形な項を取り入れたシミュレーションを行った。利得を式(\ref{eq:nonlier_gain})のように記述した。線形な項に加えて、$g_{s}$といった利得飽和の効果を取り入れている。図\ref{fig:fig_1_1_GS_pulse}にこの時のパルス生成中のキャリア密度、光子密度および利得の時間変化を表す。時刻0で2psのインパルス励起行った時の光の時間波形赤の実線と破線、キャリア密度を緑の線、材料利得を青い線で表している。
\begin{eqnarray}
g(n)&=&g_{0}(n-n_{0})\left[1+\dfrac{g_{0}(n-n_{0})}{g_{s}}\right]^{-1}\\
\label{eq:nonlier_gain}
&\simeq &\left\{
\begin{array}{ll}
 g_{0}(n-n_{0}) & n-n_{0}\ll g_{s}/g_{0}\nonumber \\
g_{s} & n-n_{0}\gg g_{s}/g_{0}\nonumber
\end{array}
\right.
\end{eqnarray}
するとIとIIの領域つまり立ち上がりの時間領域では光子密度が小さい一方でキャリア密度が大きいため、$g_{s}$が支配的に立ち上がりを決めている。IIIの領域ではキャリア密度が減少し、利得が線形になってくる領域では$\epsilon$が効いてくる。IVの領域ではキャリア密度も光子密度も小さくなっているため減衰の速さは光子の共振器寿命$\tau_{p}$によってきまる。


このように利得スイッチングには興味深い非線形性が含まれており詳細に理解を進めることは半導体レーザーそのものの理解にも繋がりうる。

\clearpage
\subsubsection{利得スイッチング光パルスの短パルス化}


\begin{figure}[h]
	\centering
	\includegraphics[width=15cm]{figure/fig_1_1_GS_pulse.png}
	\caption{パルス生成中のキャリア密度、光子密度、利得gの時間変化\cite{ref_1_1_GS}}
	\label{fig:fig_1_1_GS_pulse}
\end{figure}

\clearpage
\subsection{InGaAs高利得材料}

\begin{figure}[h]
	\centering
	\includegraphics[width=15cm]{figure/fig_1_1_lattice_constance.png}
	\caption{格子定数}
	\label{fig:fig_latice_constancce}
\end{figure}

\begin{figure}[h]
	\centering
	\includegraphics[width=15cm]{figure/fig_1_1_lattice_strain.png}
	\caption{歪み補償}
	\label{fig:fig_1_1_GS_lattice_strain}
\end{figure}
InGaAs
\clearpage
\section{本研究の目的}
電流注入により短いパルスを達成すること?
新しい構造を作ったからそれを図ること?
	%序論
\chapter{試料構造と測定方法}
\label{chap:simulation}

\section{はじめに}

\section{試料作製}
\subsection{試料構造}
\subsection{ブロードコンタクトレーザー}
\subsection{リッジ導波路型レーザー}
\subsection{マウント(ダイボンディング??)}
\section{測定方法}
本研究ではエピウエハの品質評価のための測定と利得スイッチング動作を起こしデバイスの高速特性を評価するための測定を行った。
\subsection{IL}
\begin{figure}
	\includegraphics{スクリーンショット_2018-12-01 16.51.41.png}
\end{figure}
\subsection{電流注入利得スイッチング実験}
		%試料構造と測定方法
% !TEX root = main.tex
\chapter{実験結果}
本章では3.1節ではブロードコンタクトレーザー、3.2節ではリッジ導波路型レーザーへの電流注入実験についての結果を報告する。
\section{ブロードコンタクトレーザー試料に関する測定結果}%=====================================
ブロードコンタクトレーザーへ定常電流を流してILカーブを得る実験を行った。様々な共振器長$L$、電極パッド幅$w$の試料に対して実験を行うことでウエハの基本的な物性パラメータを見積もることが目的である。

具体的には発振閾値電流$I_{\rm{th}}$を測定することに加えて、発振時の印可電流の増分に対する光出力の増大から発光量子効率(微分外部量子効率)を見積もることが目的である。
また発振閾値電流密度を算出するためにデバイス内の電流の広がりを見積もった。
3周期ウエハと10周期ウエハごとに節を分けている。
\subsection{3周期歪量子井戸ブロードコンタクトレーザー}%===============================
3周期量子井戸ブロードコンタクトレーザーの結果を示す。図\ref{fig:fig_3_1_3QW_broacdcontact_IL}(a)縦軸に発光強度(片方の端面)、横軸に電流をとったILカーブの結果を示す。また\ref{fig:fig_3_1_3QW_broacdcontact_IL}(b)は縦軸に試料にかかっている電圧、横軸に電流をとったIVカーブの結果である。共振器長$L$が$L$=500,1000,2000\si{ \micro\metre}の結果をプロットした。代表としてパッド幅$w$=50\si{ \micro\metre}の結果をプロットした。印可電流は1\si{ \micro\metre}パルスを2 ms繰り返し周期で流しており、デューティー比は1:2000である。

\begin{figure}[h]
	\centering
	\includegraphics[width=15cm]{figure/fig_3_1_3QW_broadcontact_IL.png}
		\caption{3周期歪量子井戸ブロードコンタクトレーザーのILカーブとIVカーブ}
		\label{fig:fig_3_1_3QW_broacdcontact_IL}
\end{figure}

(a)を見ると各デバイスにおいて光出力強度が電流値を上げてくと増加していき、ある電流値を超えると発振が始まり発光強度が急激に増加することがわかる。その電流値を発振しているときのILカーブを直線フィッティングすることで求めた。フィッティング直線の$x$切片を発振閾値電流$I_{\rm{th}}$とした。またフィッティング直線の傾きを発振時のスロープ効率$\Delta P/\Delta I$とした。スロープ効率はフィッティングの傾きにデューティー比をかけ、共振器の両端面からの放出を考慮して2倍にして算出した。
典型的な値として、図\ref{fig:fig_3_1_3QW_broacdcontact_IL}(a)のILカーブの値を表\ref{table:table_3_1_3QW_broadcontact}に示す。表では
\begin{table}[h]
  \caption{3周期ブロードコンタクトレーザーの閾値電流}
  \label{table:table_3_1_3QW_broadcontact}
  \centering
  \begin{tabular}{ccc}
    \hline
    共振器長$L$(um)  & 閾値電流$I_{th}$ (mA)  & Slope 2$\Delta P/\Delta I$ (W/A) \\
    \hline \hline
     500& 187&  0.83  \\
    1000& 234& 0.51\\
    2000& 450&0.37\\
       \hline
  \end{tabular}
\end{table}



\ref{fig:fig_3_1_3QW_broacdcontact_IL}(b)を見ると各デバイスにおいて電流が流れ始めるのが1 V付近からでありダイオード特性が見られる。また共振器長$L$が長いほど同じ電流に対する電圧が低い。これは共振器長$L$に比例して電流が流れる面積が大きくなるためデバイスの抵抗値が小さくなっているためである。


次に様々なパッド幅に対して見積もった発振閾値電流$I_{\rm{th}}$の結果を図\ref{fig:fig_3_1_3QW_broadcontact_Ith}(a)に示す。発振閾値電流$I_{\rm{th}}$、横軸が電極パッド幅$w$である。\ref{fig:fig_3_1_3QW_broadcontact_Ith}(b)は発振時の発光効率$2 \Delta P/\Delta I$である。

図\ref{fig:fig_3_1_3QW_broadcontact_Ith}(a)を見ると、パッド幅$w$が50\si{ \micro\metre}より大きい領域では閾値電流$I_{\rm{th}}$はパッド幅$w$に対して線形に増加していることがわかる。一方パッド幅$w$が小さい領域では線形に変化していない。

この原因は電流がパッド幅$w$に対して無視できないほど広がってしまっているためだと考えられる。電流広がりについてはフィッティングから見積もった。詳しくは3.1.3節で述べる。

図\ref{fig:fig_3_1_3QW_broadcontact_Ith}(b)を見るとそれぞれの共振器長で概ね横ばいの値を持っている。$\Delta P/\Delta I$はパッド幅$w$に依存しないことがわかる。幅wは光の増幅を受ける方向とは関係がなく、直感と一致する。

\begin{figure}[h]
	\centering
	\includegraphics[width=15cm]{figure/fig_3_1_3QW_broadcontact_Ith.png}
		\caption{3周期歪量子井戸ブロードコンタクトレーザーの閾値電流と発光効率}
		\label{fig:fig_3_1_3QW_broadcontact_Ith}
\end{figure}
\clearpage
\subsection{10周期歪補償量子井戸ブロードコンタクトレーザー}%===============================
次に10周期歪補償量子井戸ブロードコンタクトレーザーについての結果を示す。図\ref{fig:fig_3_1_10QW_broadcontact_IL}(a)にILカーブ、(b)にIVカーブを示す。$w$=50 \si{\micro\metre}を代表としてプロットした。色分けは共振器長の違いを表す。電流は2 \si{\micro s}パルスを2 ms繰り返し周期で印可した。デューティー比は1:1000である。
\begin{figure}[h]
	\centering
	\includegraphics[width=15cm]{figure/fig_3_1_10QW_broadcontact_IL.png}
		\caption{10周期歪補償ブロードコンタクトレーザーのILカーブとIVカーブ}
		\label{fig:fig_3_1_10QW_broadcontact_IL}
\end{figure}

典型的な値として図
\ref{fig:fig_3_1_10QW_broadcontact_IL}(a)のILカーブフィッティング結果の値を表\ref{table:table_3_1_10QW_broadcontact}に示す。傾きはデューティー比1:1000と両端面からの発光を換算していることに注意されたい。
\begin{table}[h]
  \caption{10周期歪補償ブロードコンタクトレーザーの閾値電流}
  \label{table:table_3_1_10QW_broadcontact}
  \centering
  \begin{tabular}{ccc}
    \hline
    共振器長$L$(\si{\micro\metre})  & 閾値電流$I_{th}$ (mA)  & Slope 2$\Delta P/\Delta I$ (W/A) \\
    \hline \hline
     500& 212&  0.64  \\
    1000& 363& 0.42\\
    2000& 501&0.27\\
       \hline
  \end{tabular}
\end{table}

次に\adsp02{図}\ref{fig:fig_3_1_3QW_broadcontact_Ith} (a)に\ref{fig:fig_3_1_10QW_broadcontact_IL} (a)のILカーブの発振時の直線フィッティング結果から求めた閾値電流$I_{\rm{th}}$および(b)スロープ効率2 $\Delta P/\Delta I$をパッド幅$w$に対してプロットした。
\begin{figure}[h]
	\centering
	\includegraphics[width=15cm]{figure/fig_3_1_10QW_broadcontact_Ith.png}
		\caption{10MQWのIL結果}
		\label{fig:fig_3_1_10QW_broadcontact_Ith}
\end{figure}

\subsection{電流広がりに関する考察}%==================
3.1.1節と3.1.2節でILカーブから3周期歪量子井戸ブロードコンタクトレーザーと10周期歪補償ブロードコンタクトレーザーについて閾値電流$I_{\rm{th}}$を見積もった。閾値電流密度$J_{\rm{th}}$は半導体レーザーの性能を表す指標にもなるパラメータである。
レーザーの基本的な特性を知る上で閾値電流密度が大切なパラメータであるからである。発振閾値電流$I_{\rm{th}}$を電流が流れた面積で割ることで閾値電流密度$J_{\rm{th}}$が求まる。

通常閾値電流$I_{\rm{th}}$は電流を流す面積に比例して大きくなる。面積とは電極パッド幅$w$と共振器長$L$の積で表される。つまり$I_{\rm{th}}$は$w$に対して線形に増加するはずである。しかし図\ref{fig:fig_3_1_3QW_broacdcontact_IL} (a)や\ref{fig:fig_3_1_10QW_broadcontact_IL}(a) を見るとそうなっていない。そこで発振閾値電流$I_{\rm{th}}$が線形に増加する領域を直線フィッティングし、その直線の$x$切片を含めたパッド幅を有効的なパッド幅と考えて閾値電流密度を算出した。まずは有効パッド幅を見積もった。フィッティング関数の$x$切片の絶対値が実質的なパッド幅の増分$w'$である。その値を表\ref{table:table_3QW_broadcontact_w_eff}と表\ref{table:table_10QW_broadcontact_w_eff}に示した。
\begin{table}[h]
  \caption{3周期歪量子井戸ブロードコンタクトレーザーの電流広がり}
  \label{table:table_3QW_broadcontact_w_eff}
  \centering
  \begin{tabular}{cc}
    \hline
    共振器長$L$ (\si{\micro\metre})  & パッド幅の増分(電流の広がり) $w'$ (\si{\micro\metre})   \\
    \hline \hline
     500 & 65.8  \\
    1000  & 54.1 \\
    2000  & 58.7 \\ 
    \hline
  \end{tabular}
\end{table}

\begin{table}[h]
  \caption{10周期歪補償ブロードコンタクトレーザーの電流広がり}
  \label{table:table_10QW_broadcontact_w_eff}
  \centering
  \begin{tabular}{cc}
    \hline
    共振器長$L$ (\si{\micro\metre})  & パッド幅の増分(電流の広がり) $w'$ (\si{\micro\metre})   \\
    \hline \hline
     500 & 51.1  \\
    1000  & 39.5 \\
    2000  & 25.7 \\ 
    \hline
  \end{tabular}
\end{table}

3周期歪量子井戸レーザーでは58 \si{\micro\metre}から65 \si{\micro\metre}程度の広がりであることが見積もられた。10周期歪補償量子井戸レーザーでは25\si{ \micro\metre}から51\si{ \micro\metre}の広がりであると見積もられた。10周期歪補償値量子井戸レーザーでは値ののばらつきが大きく$L$=500 \si{\micro\metre}の$w'$と$L$=2000 \si{\micro\metre}の$w'$を比較すると2倍程度異なってしまっている。これは共振器の長い試料について、$w$が大きい試料に対しての実験結果がないため、$w'$の見積もりが小さくなってしまったためであると考えられる。さらに大電流を流して発振させる実験を行うことが必要である。


この表の値$w'$と閾値電流$I_{\rm{th}}$(mA)から式(\ref{eq:Jth})の関係を用いて閾値電流密度$J_{\rm{th}} \rm{(kA/cm^2)}$を算出した。
\begin{eqnarray}
J_{\rm{th}}=\dfrac{I_{\rm{th}}}{(w+w')L}
\label{eq:Jth}
\end{eqnarray}

その結果を示す。図\ref{fig:fig_3_1_3QW_broadcontact_Jth}に3周期歪量子井戸ブロードコンタクトレーザーの閾値電流密度、図\ref{fig:fig_3_1_10QW_broadcontact_Jth}に10周期歪補償量子井戸ブロードコンタクトレーザーの閾値電流密度をプロットした。

\begin{figure}[h]
	\centering
	\includegraphics[width=10cm]{figure/fig_3_1_3QW_broadcontact_Jth.png}
		\caption{3周期歪量子井戸ブロードコンタクトレーザーの閾値電流密度}
		\label{fig:fig_3_1_3QW_broadcontact_Jth}
\end{figure}

\begin{figure}[ht]
	\centering
	\includegraphics[width=10cm]{figure/fig_3_1_10QW_broadcontact_Jth.png}
		\caption{10周期歪補償量子井戸ブロードコンタクトレーザの閾値電流密度}
		\label{fig:fig_3_1_10QW_broadcontact_Jth}
\end{figure}
3周期歪量子井戸ブロードコンタクトレーザーでは$0.20\sim 0.35  \rm{ kA/cm^{2}}$、10周期歪補償ブロードコンタクトレーザーでは$0.40 \rm{ kA/cm^{2}}$程度と見積もられた。

3周期歪量子井戸ブロードコンタクトレーザーでは $w$が50 \si{\micro\metre}より大きい領域で、共振器長が長くなるほど閾値電流密度$J_{\rm{th}}$が小さくなることがわかる。
%これは共振器長が長くなるほどミラーロスに対する内部ロスが大きくなっていき、発振が起こりにくくなっているためである。

一方10周期歪補償量子井戸ブロードコンタクトレーザーではL=2000 \si{\micro\metre}の値が他の2つに比べて大きくなってしまっている。これは$w'$が小さく見積もられており、$J_{\rm{th}}$が大きく見積もられたためだと考えられる。$w'$の解析に用いたプロット点数が少ないことが原因である。

パッド幅$w$の広がり$w'$が数十\si{\micro\metre}とパッド幅に対して優位なほど広がっているという推察を得たが、この原因としてウエハの結晶構造が考えられる。図\ref{fig:fig_2_1_wafer_structure}のエピウエハ構造において活性層の上のSCH層の上に100 nm厚のp-$\rm{In_{0.485}Ga_{0.515}P}$層が挿入されている。このp-$\rm{In_{0.485}Ga_{0.515}P}$のバンドギャップは1.891 eVは上下を挟んでいるGaAsのバンドギャップ1.424 eVに比べて0.467 eV大きい。このバンドギャップの差によりキャリアが拡散され結晶面内へ広がってしまったと考えられる。

\clearpage
\subsection{外部量子効率、内部量子効率と吸収係数の計算}%=============
次にILカーブの発振時の傾きに相当するスロープ効率$\Delta P/\Delta I$から試料の内部微分量子効率$\eta_{i}$および吸収係数$\alpha$を算出した。まずはスロープ効率$2 \Delta P/\Delta I \rm{ (W/A)}$から外部微分量子効率$\eta_{\rm{d}}$を算出した。式({\ref{eq:eta_d})の関係を用いた。
\begin{eqnarray}
\eta_{\rm{d}}=\dfrac{e}{h\nu}2\dfrac{\Delta P}{\Delta I} 
\label{eq:eta_d}
\end{eqnarray}
eは電気素量、hはプランク定数、$\nu$は発振周波数であり、1050 nmとした。$\eta_{\rm{d}}$はキャリアの注入数に対する取り出せる光子の数の割合である。結果を図\ref{fig:fig_3_1_3QW_broadcontact_id}に示す。縦軸を$\eta_{\rm{d}}$横軸をパッド幅$w$としてプロットした。色分けが共振器長の違いを表している。$L$=500 \si{\micro\metre}では0.7程度、$L$=1000         \si{\micro\metre}では0.4程度、L=400 \si{\micro\metre}では0.3程度の値を持っている。
\begin{figure}[h]
	\centering
	\includegraphics[width=10cm]{figure/fig_3_1_3QW_broadcontact_id.png}
	\caption{3周期歪量子井戸ブロードコンタクトレーザーの外部量子効率}
	\label{fig:fig_3_1_3QW_broadcontact_id}
\end{figure}

ここで$\eta_{d}$は共振器内での全発光にしめる共振器損失の割合であるから
\begin{eqnarray}
\eta_{d}=\eta_{int}\dfrac{\alpha_{m}}{\alpha_{int} +\alpha_{m}}
\end{eqnarray}
である。$\alpha_{int}$は共振器内の平均の内部損失、Rは共振器の端面での反射率、$\eta_{\rm{int}}$は内部微分量子効率である。
$\eta_{\rm{d}}$は共振器長$L$を用いて式(\ref{eq:eta_inverse})と書き表される。
\begin{eqnarray}
\dfrac{1}{\eta_{\rm{d}}}=\dfrac{\alpha_{\rm{int}}}{\rm{ln}(1/R)\eta_{int}}L+\dfrac{1}{\eta_{\rm{int}}}
\label{eq:eta_inverse}
\end{eqnarray}



RはGaAsの屈折率と空気の屈折率の差から0.32と仮定した。見積もった$\eta_{\rm{d}}$の逆数を共振器長に対してプロットし直線フィッティングを行なった。これを図\ref{fig:fig_3_1_3QW_broadcontact_id_inverse}に示す。横軸は共振器長$L$、縦軸に外部微分量子効率の逆数$1/\eta_{\rm{d}}$である。例としてパッド幅$w$=100 \si{\micro\metre}の結果を示す。式(\ref{eq:eta_inverse})よりこのフィッティング直線のy切片から内部量子効率$\eta_{\rm{int}}$を見積もると$\eta_{\rm{int}}$=0.96と計算できる。また、直線の傾きから内部損失$\alpha_{\rm{int}}$を見積もると$\alpha_{\rm{int}}$=11.8 /cmと計算できた。
\begin{figure}[h]
	\centering
	\includegraphics[width=10cm]{figure/fig_3_1_3QW_broadcontact_id_inverse.png}
	\caption{3周期歪量子井戸レーザーの外部量子効率の逆数}
	\label{fig:fig_3_1_3QW_broadcontact_id_inverse}
\end{figure}



\newpage

同様の解析を10周期歪補償量子井戸ブロードコンタクトレーザーの結果についても行った。図\ref{fig:fig_3_1_10QW_broadcontact_id}に外部微分量子効率、図\ref{fig:fig_3_1_10QW_broadcontact_id_inverse}\adsp02{に外部微分量子効率の逆数の共振器長依存性を示す。横軸は共振器長$L$、縦軸は外部微分量子効率の逆数$1/\eta_{\rm{d}}$である。}$w$=50\si{ \micro\metre} の結果を示している。10周期に関しては$\eta_{\rm{int}}$=0.94、$\alpha_{\rm{int}}$=18.0 (/cm)となった。
\begin{figure}[h]
	\centering
	\includegraphics[width=10cm]{figure/fig_3_1_10QW_broadcontact_id.png}
	\caption{10QW外部量子効率}
	\label{fig:fig_3_1_10QW_broadcontact_id}
\end{figure}

\begin{figure}[h]
	\centering
	\includegraphics[width=10cm]{figure/fig_3_1_10QW_broadcontact_id_inverse.png}
	\caption{10QW外部量子効率の逆数}
	\label{fig:fig_3_1_10QW_broadcontact_id_inverse}
\end{figure}

\clearpage
\subsection{透明電流密度の見積もり}

次に透明電流密度$J_{0}$の見積もりを行った。共振器内の正味の利得$g_{\rm{net}}$は
\begin{eqnarray}
g_{\rm{net}}=\Gamma G-\alpha_{int}-\alpha_{m}
\label{eq:eta_j0}
\end{eqnarray}
と書ける。線形利得$G=g_{0}(J-J_{0})$を仮定すると閾値電流密度は
\begin{eqnarray}
J_{th}=J_{0}+\dfrac{\alpha_{int}}{\Gamma g_{0}}+\dfrac{1}{\Gamma g_{0}}\rm{ln}\left(\dfrac{1}{R}\right)\dfrac{1}{L}
\label{eq:j_th}
\end{eqnarray}
と書ける。この式にしたがうと$J_{\rm{th}}$は$1/L$に比例する。図\ref{fig:fig_3_1_3QW_broadcontact_j0}に$1/L$に対して$J_{\rm{th}}$をプロットした。色分けは共振器長を表す。
\begin{figure}[h]
	\centering
	\includegraphics[width=10cm]{figure/fig_3_1_3QW_broadcontact_j0.png}
	\caption{3周期歪量子井戸ブロードコンタクトレーザーの透明電流密度の見積もり}
	\label{fig:fig_3_1_3QW_broadcontact_j0}
\end{figure}

図\ref{fig:fig_3_1_3QW_broadcontact_j0}のプロットのなかで、図\ref{fig:fig_3_1_3QW_broadcontact_Jth}の閾値電流密度をプロットした図において$J_{\rm{th}}$が一定となっている$w$が50 \si{\micro\metre}以上のプロットに対して線形フィッティングを行った。赤い直線がフィッティング直線である。フィッティング結果と式(\ref{eq:j_th})の関係を用いて$\Gamma g_{0}$と$J_{0}$を見積もると$\Gamma g_{0}=151 \rm{ kA^{-1}}$、$J_{0}=0.0782 \rm{kA/cm^{2}}$を得た。


10周期歪補償量子井戸ブロードコンタクトレーザーについても同様の解析を行った。図\ref{fig:fig_3_1_10QW_broadcontact_j0}に$J_{\rm{th}}$対$1/L$のプロットを示す。フィッティング結果から$\Gamma g_{0}$と$J_{0}$を見積もると$\Gamma g_{0}=558   \rm{kA^{-1}}$、$J_{0}=0.357 \rm{kA/cm^{2}}$を得た。
\begin{figure}[t]
	\centering
	\includegraphics[width=10cm]{figure/fig_3_1_10QW_broadcontact_j0.png}
	\caption{10周期歪補償量子井戸ブロードコンタクトレーザーの透明電流密度の見積もり}
	\label{fig:fig_3_1_10QW_broadcontact_j0}
\end{figure}
\newpage
\subsection{ブロードコンタクトレーザーに対する電流注入実験のまとめと考察}

実験結果をまとめると3周期歪量子井戸ブロードコンタクトレーザーと10周期歪補償ブロードコンタクトレーザーに関して以下の表のようになる。
\begin{table}[h]
  \caption{ブロードコンタクトレーザーの結果まとめ}
  \label{table:table_I0}
  \centering
  \begin{tabular}{lcccc}
    \hline
    試料   &  内部損失$\alpha_{int} \rm{ /cm}$&内部量子効率$\eta_{int} $&透明電流密度 $J_{0} \rm{  kA/cm^2}$  &$\Gamma g_{0}$ /kA\\
    \hline \hline
     3周期 &   11.8 &0.96&0.0782 & 151\\
    10周期s  & 18.0 &0.94&0.357&558\\
    \hline
  \end{tabular}
\end{table}

内部損失は10周期試料の方が多くなった。
3周期試料と10周期試料を比較したとき透明電流密度は10周期試料が4.6倍となった。層数を増加させた効果が見えた。層数の比(活性層の厚さの比に等しい)が3.3倍であることを考えると式(\ref{eq:Mj0})で予想された透明電流密度の比よりも大きくなっている。また微分\adsp02{モード}利得係数$\Gamma g_{0}$の比は558/151=3.7となった。
モード利得$\Gamma G=\Gamma g_{0}(J-J_{0})$を図\ref{fig:fig_3_1_broadcontact_modal_gain}にプロットした。

\begin{figure}[t]
	\centering
	\includegraphics[width=10cm]{figure/fig_3_1_broadcontact_modal_gain.png}
	\caption{モード利得の電流密度依存性}
	\label{fig:fig_3_1_broadcontact_modal_gain}
\end{figure}
これを見ると0.5$ \rm{kA/cm^2}$ 以下では層数の少ない3周期試料の方がモード利得は大きく、それ以降では10周期試料の方が高いモード利得が得られることがわかる。この振る舞いは図	\ref{fig:fig_gain_mode}の理論計算により求められたモード利得の振る舞いと一致している。
\clearpage
\section{リッジ導波路型レーザーに関する実験結果}%===================
ブロードコンタクトレーザーを用い、半導体レーザーの基本的な特性を調べた。次に完成したデバイスとしてのリッジ導波路型レーザーに短パルス電流を印可し、利得スイッチング動作を行った。
\subsection{定常電流の結果}

利得スイッチング動作を行う前にまずは発振するか確かめること、および閾値電流を見積もることを目的として定常電流による標準的なデバイス評価実験を行なった。方法はブロードコンタクトレーザーの評価測定と同じである。
\subsubsection{3周期歪量子井戸リッジ導波路型レーザーの結果}
まずは3周期歪量子井戸リッジ導波路型レーザーの結果を示す。2 \si{\micro s}パルスを2 ms周期で印可した。デューティー比は1:1000である。
まずは3周期歪量子井戸リッジ導波路型レーザーの結果を図\ref{fig:fig_3_2_3QW_ridge_IL}に示す。図\ref{fig:fig_3_2_3QW_ridge_IL}(a)はILカーブ、(b)はIVカーブである。色分けは共振器長$L$の違いを表す。

\begin{figure}[h]
	\centering
	\includegraphics[width=15cm]{figure/fig_3_2_3QW_ridge_IL.png}
		\caption{3周期歪量子井戸リッジ導波路型レーザーのILカーブおよびIVカーブ}
		\label{fig:fig_3_2_3QW_ridge_IL}
\end{figure}


次にILカーブから見積もった閾値電流$I_{\rm{th}}$、$J_{\rm{th}}$と閾値電流密度を図\ref{fig:fig_3_2_3QW_ridge_Ith}に示す。図中の丸プロットが閾値電流$I_{\rm{th}}$であり左の軸に対してのプロットした。十字プロットは閾値電流を共振器長とリッジ幅の積で割った値の閾値電流密度$J_{\rm{th}}$であり右の軸に対してのプロットである。横軸は共振器長である。色分けはリッジ幅の違いを表している。紫色がリッジ幅1.5 \si{\micro\metre}、黄色が2.5 \si{\micro\metre}である。

これを見ると閾値電流は共振器長に対して概ね線形に増加しており\delsp02{最短では}\adsp02{最小の閾値電流は$L$=100 \si{\micro\metre}のときであり}50 mA程度となっている。またリッジ幅を1.5 \si{\micro\metre} 、 2.5 \si{\micro\metre}と変えても閾値電流に差が見られていない。これはブロードコンタクトレーザーで示唆されたように電流が広がってしまい有効的なリッジ幅は実際のリッジ幅よりも広いと考えられる。したがって閾値電流密度を見積もることは難しい。

閾値電流密度は$L$が300 \si{\micro\metre}より大きい共振器長において概ね横ばいとなっており10から20 $\rm{kA/cm^2}$の値を持っている。またリッジ幅による差異は単に同程度の閾値電流を異なるリッジ幅で割ったためである。


\begin{figure}[h]
	\centering
	\includegraphics[width=10cm]{figure/fig_3_2_3QW_ridge_Ith.png}
		\caption{3周期歪量子井戸リッジ導波路型レーザーの$I_{\rm{th}}$、$J_{\rm{th}}$}
		\label{fig:fig_3_2_3QW_ridge_Ith}
\end{figure}
次にILカーブの発振領域の発光効率$2 \Delta P/\Delta I$および外部微分量子効率$\eta_{\rm{d}}$を共振器長に対してプロットした結果を図\ref{fig:fig_3_2_3QW_ridge_slope}に示す。発光効率$2 \Delta P/\Delta I$は0.03から0.87 W/A の値となった。また外部微分量子効率は0.026から0.87の値を持っている。
%L=100だけ低くなる原因ある?
\begin{figure}[h]
	\centering
	\includegraphics[width=10cm]{figure/fig_3_2_3QW_ridge_slope.png}
		\caption{3周期歪量子井戸リッジ導波路型レーザーのスロープおよび外部微分量子効率}
		\label{fig:fig_3_2_3QW_ridge_slope}
\end{figure}

\clearpage
\subsubsection{10周期歪補償量子井戸リッジ導波路レーザー}
次に10周期歪補償量子井戸リッジ導波路レーザーの結果を示す。図\ref{fig:fig_3_2_10QW_ridge_IL}(a)にILカーブ、(b)にIVカーブを示す。

図\ref{fig:fig_3_2_10QW_ridge_IL}(a)を見るとそれぞれの共振器長において発振したことがわかる。$L$= 400\si{\micro\metre}の赤い線を見ると$I$= 200mA付近からどの試料においても発光量が下がってきている。
\begin{figure}[h]
	\centering
	\includegraphics[width=15cm]{figure/fig_3_2_10QW_ridge_IL.png}
		\caption{10周期歪量子井戸リッジ導波路型レーザーのILカーブおよびIVカーブ}
		\label{fig:fig_3_2_10QW_ridge_IL}
\end{figure}
次にILカーブから閾値電流$I_{\rm{th}}$と閾値電流密度$J_{\rm{th}}$を算出した。その結果を図\ref{fig:fig_3_2_10QW_ridge_Ith}に示す。閾値電流は共振器長に対して線形に増加しており最小で90 mA程度となった。色分けはリッジ幅の違いを表すが、3周期試料と同様に閾値電流においてリッジ幅の際は見られない。閾値電流密度は10から20 $\rm{kA/cm^{2}}$程度となったが電流広がりの影響を考えていないため正しく見積もることはできていない。
\begin{figure}[h]
	\centering
	\includegraphics[width=10cm]{figure/fig_3_2_10QW_ridge_Ith.png}
		\caption{10周期歪補償量子井戸リッジ導波路型レーザーの閾値電流と閾値電流密度}
		\label{fig:fig_3_2_10QW_ridge_Ith}
\end{figure}
ILカーブの発振時の傾きから見積もったスロープ効率$2\Delta P/\Delta I$および外部微分量子効率$\eta_{\rm{d}}$を図\ref{fig:fig_3_2_10QW_ridge_slope}に示す。$\eta_{d}$は発振波長1030 nmとして計算した。通常$\eta_{d}$は共振器長に対しては減少するはずであるが(式(\ref{eq:eta_inverse}による))、 $L$=300 \si{\micro\metre},  400  \si{\micro\metre}では$L$の増加に対して減少が見られない。図\ref{fig:fig_3_2_10QW_ridge_IL}(a)を見てもこの2種類の共振器長に関してはILカーブが曲がり発光量が減少していることがわかる。


\begin{figure}[ht]
	\centering
	\includegraphics[width=10cm]{figure/fig_3_2_10QW_ridge_slope.png}
		\caption{10周期歪補償量子井戸リッジ導波路型レーザーのスロープおよび外部微分量子効率}
		\label{fig:fig_3_2_10QW_ridge_slope}
\end{figure}

\subsection{リッジ導波路型レーザーへの定常電流注入実験の結果まとめ}
\subsubsection{3周期歪量子井戸試料と10周期歪補償量子井戸試料の比較}
3周期試料と10周期試料を比較すると閾値電流は同程度外部微分量子効率は\delsp02{1}\adsp02{3}周期試料の方が高いという結果を得た。

また発振する前の発光強度を見てみると10周期試料の方が大きい\delsp02{ことがわかる}。自然放出光の強度が10周期試料の方が大きいことがわかる。
\subsubsection{10周期歪補償リッジ導波路型レーザーのILカーブのドループについて}
ここで10周期歪補償量子井戸リッジ導波路型レーザーのILカーブにおいて観測された発光の飽和および\adsp02{ドループ(注入増に対する発光強度減)}\delsp02{減衰}について述べる。このような発光量の飽和現象の原因としてはデバイスの温度上昇やそれに伴う吸収係数の増大、利得飽和(レート方程式における$\epsilon$の効果)、反射端面の光学損傷、空間のホールバーニング効果などが挙げられる。

これらの中から実験的に原因を特定するためには、発光の遠視野像を撮像することや活性層ではなくp側のGaAs層での発光強度を観察することが良いと考えられる。特に先の実験からInGaP層が障壁の役割をしていると想像されているため、その上下のGaAs層で発光が起こっている可能性が高く、その注入電流に対するGaAs発光強度を測定することができればデバイス内部のキャリアの分布が決められるのではないかと考えられる。
%ここでキャリアの再結合過程を考えると式(\ref{eq:tau_r})の非発光再結合には
%\begin{equation}
%\dfrac{1}{\tau_{nr}}(n)=A_{1}+A_{2}n+A_{3}n^2
%\end{equation}
%とキャリア密度nの二乗に比例するようなオージェ再結合が起こることが知られている。
%ILカーブを見るとこの中で共振器長に対して依存性が
%空間ホールバーニングはモードを確認すれは良い,filamentation(繊維化)

\clearpage
\subsection{短パルス電流注入の結果}

次にリッジ導波路型レーザーに関して1 ns矩形波電気パルスを印可し電流注入利得スイッチング実験を行った。そのILカーブと時間波形を示す。
\subsection{ILカーブ}
短パルス駆動時のILカーブを示す。その際電流に換算することが困難であったため、横軸はパルスの電圧である。
\subsubsection{3周期歪量子井戸レーザーのILカーブ}
図\ref{fig:fig_3_2_3QW_ridge_GS_power}に3周期歪量子井戸リッジ導波路型レーザーの短パルス駆動時のILカーブを示す。励起パルスのパルス幅は1 nsである。3つの異なる共振器長において発振が確認できた。共振器長はL=100 \si{\micro\metre}、200 \si{\micro\metre}、300 \si{\micro\metre}である。試料はレーザーバーの状態のものを用いた。\adsp02{1 cm程度芯線をむき出しにした同軸ケーブルを試料から5 mm程度の場所に固定し、芯線とp電極を金線でワイヤリングを行った。作業の都合上$L$=200 \si{\micro\metre}に関してワイヤーの長さは他の試料よりも2 mm程度長くなった。同軸ケーブルののグランド側は試料が乗っている約2 cm角の銅板に密着させた。}\delsp02{むき出しにしたものを試料の近く1cm程度($L$=200 umは3 cm程度と遠い)まで近づけ、芯線とレーザーの電極を金線でワイヤリングを行った。したがって回路全体の特性}インピーダンスにマッチ\delsp02{しているか定かでない。}\adsp02{のためのマッチング抵抗の付加などを行っていない。}

\begin{figure}[h]
	\centering
	\includegraphics[width=10cm]{figure/fig_3_2_3QW_ridge_GS_power.png}
		\caption{3周期歪量子井戸 短パルス駆動時のILカーブ}
		\label{fig:fig_3_2_3QW_ridge_GS_power}
\end{figure}


\newpage
\subsubsection{10周期歪補償量子井戸レーザーのILカーブ}
次に図\ref{fig:fig_3_2_10QW_ridge_GS_power}に10周期歪補償量子井戸リッジ導波路型レーザーの短パルス駆動時のILカーブを示す。共振器長は$L$=300 \si{\micro\metre}、400 \si{\micro\metre}、500 \si{\micro\metre}である。10周期試料は全て\adsp02{1つずつに分離、チップ化しTO-}CANタイプの\adsp02{のキャリアにマウントした}試料である。\delsp02{インピーダンスマッチが取れており、高速電気パルスが形状を崩さず入っているとみなしている。}\adsp02{TO-CANのp側の足をSMAコネクタの芯線とAuSnはんだを用いて導通を取り(距離は0 )、n側の足は5 mm程度の金線空中配線によりSMAコネクタのグランドと繋げた。}

図\ref{fig:fig_3_2_10QW_ridge_GS_power}を見ると3つの異なる共振器長の試料に対して発振が確認できた。
%発振閾値書くか?
\begin{figure}[h]
	\centering
	\includegraphics[width=10cm]{figure/fig_3_2_10QW_ridge_GS_power.png}
		\caption{10歪補償量子井戸リッジ導波路型レーザーの短パルス駆動時のILカーブ}
		\label{fig:fig_3_2_10QW_ridge_GS_power}
\end{figure}
\clearpage
\subsection{3周期歪量子井戸試料の利得スイッチング動作}%===============================
次にフォトダイオードで光を検出し高速オシロスコープで電気信号をモニタした時間波形を示す。図\ref{fig:fig_3_2_3QW_ridge_L100_GS}(a)に3周期歪量子井戸レーザーの共振器長$L$=100 \si{\micro\metre}試料の利得スイッチング動作の時間波形を示す。励起強度を変えた実験結果を示す。図\ref{fig:fig_3_2_3QW_ridge_L100_GS}(b)には強度を規格化したプロットをしめす。(a)を見ると励起強度を\delsp02{あげる}\adsp02{増大させる}にしたがってピーク強度が高くなって行くが途中で頭打ちになっている。(b)を見ると21 V程度までは励起強度の増加につれて立ち上がりが早くなっているがそれより強励起ではわずかに遅くなって\delsp02{いって}いる。
\begin{figure}[h]
	\centering
	\includegraphics[width=15cm]{figure/fig_3_2_3QW_ridge_L100_GS.png}
		\caption{3周期歪量子井戸レーザー $L$=100 \si{\micro\metre} の利得スイッチング光パルスの時間波形}
		\label{fig:fig_3_2_3QW_ridge_L100_GS}
\end{figure}


図\ref{fig:fig_3_2_3QW_ridge_L200_GS}には$L$=200 \si{\micro\metre}試料の結果を示す。(a)を見ると励起強度を増加させるにしたがって1つめの光パルス強度は途中までは増加するもののあるところから減衰することがわかる。また途中から第2の光パルスが見られる。電流注入利得スイッチングに特有の緩和振動である。(b)を見ると光パルスの立ち上がりは励起強度とともに遅くなっていく様子が見られる。
%2番目のパルスの地上がり時間からなんかわからんの?それが緩和振動なのかどうか


\begin{figure}[h]
	\centering
	\includegraphics[width=15cm]{figure/fig_3_2_3QW_ridge_L200_GS.png}
		\caption{3周期歪量子井戸レーザー $L$=200 \si{\micro\metre} の利得スイッチング光パルスの時間波形}
		\label{fig:fig_3_2_3QW_ridge_L200_GS}
\end{figure}



図\ref{fig:fig_3_2_3QW_ridge_L300_GS}には$L$=300 \si{\micro\metre}試料の結果を示す。(a)の青い矢印は第1ピーク\delsp02{の場所}\adsp02{位置}の移り変わりを表している。(a)を見ると第1パルスは一度極大値を持ったのち遅くなっている。また途中から第2パルスが見られるようになり第2パルスの方が第1パルスよりも大きくなる場合が見られる。(b)を見ると励起強度を上げていくと20.0 Vまではシングルパルスであることがわかる。20.0 Vで立ち上がり時間が遅くなり、それ以上の励起強度では複数のピークを持ったまま立ち上がり時間が早くなっていく様子がわかる。長い電流パルスの影響による緩和振動であると考えられる。
\begin{figure}[h]
	\centering
	\includegraphics[width=15cm]{figure/fig_3_2_3QW_ridge_L300_GS.png}
		\caption{3周期歪量子井戸レーザー$ L$=300 \si{\micro\metre} の利得スイッチング光パルスの時間波形}
		\label{fig:fig_3_2_3QW_ridge_L300_GS}
\end{figure}


$L$=100 \si{\micro\metre}、200 \si{\micro\metre}、$L$=300 \si{\micro\metre}で見られた立ち上がりが遅くなる現象およびは通常の利得スイッチングの動作とは異なる。その原因は励起パルスが正常に印加されていないためではないかと考えられる。どの試料も配線する際の金線の長さが長く\delsp02{、インピーダンスが大きくなってしまったため、}短い電圧パルスが形を保てなかったと同時に励起強度も低くなってしまったと推測できる。
\subsubsection{3周期歪量子井戸リッジ導波路型レーザーの利得スイッチング実験結果まとめ}
ここで利得スイッチング光パルスの第1パルスのパルス幅を示す。ここでパルス幅は半値全幅FWHMとしている。また光の検出に用いた25 GHzフォトダイオードによるパルス広がりを考慮してdeconvolutionを行った結果を示す。

図\ref{fig:fig_3_2_3QW_ridge_GS_FWHM}に3周期歪量子井戸リッジ導波路型レーザーの利得スイッチング光パルスのパルス幅を示す。縦軸FWHM、横軸が励起強度である。色分けは共振器長の違いを表す。

$L$=100 \si{\micro\metre}、200 \si{\micro\metre}では励起強度を上げるとパルス幅が長くなっている。\delsp02{これはインピーダンスマッチが取れていない事による電気パルスの変形だと考えられる。}$L$=300 \si{\micro\metre}ではパルス幅30 ps程度の値を示し。飽和が起こっている。最短パルス幅はL=300 \si{\micro m}、23.4 V印加で28.9 psであった。
\begin{figure}[ht]
	\centering
	\includegraphics[width=10cm]{figure/fig_3_2_3QW_ridge_GS_FWHM.png}
		\caption{3周期歪量子井戸リッジ導波路型レーザーの光パルス幅}
		\label{fig:fig_3_2_3QW_ridge_GS_FWHM}
\end{figure}
\clearpage
\subsection{10周期歪補償量子井戸リッジ導波路型レーザーの利得スイッチング動作}%==============================
次に10周期歪補償量子井戸試料の利得スイッチング時間波形の結果を示す。

図\ref{fig:fig_3_2_10QW_ridge_L300_GS}には共振器長300 \si{\micro\metre}の結果を示す。(a)は時間波形の生データ、(b)は規格化したデータである。(a) を見ると励起強度をあげるにしたがって第1ピーク強度は大きくなっている。(b)を見ると最初はシングルピークだった光パルスが27 Vを超えたところから緩和振動がみられ複数ピークになっている。第1ピークは励起強度とともに早く立ち上がる様子が見られる。典型的な利得スイッチング動作である。
\begin{figure}[h]
	\centering
	\includegraphics[width=15cm]{figure/fig_3_2_10QW_ridge_L300_GS.png}
		\caption{10周期歪補償量子井戸リッジ導波路型レーザー $L$=300 \si{\micro\metre} の利得スイッチング光パルスの時間波形}
		\label{fig:fig_3_2_10QW_ridge_L300_GS}
\end{figure}


図\ref{fig:fig_3_2_10QW_ridge_L400_GS}には共振器長$L$=400 \si{\micro\metre}の試料の結果を示す。(a)を見ると第1ピークは励起強度とともに大きくなっていく様子が見える。(b)を見ると励起強度を上げると緩和振動が見られるようになっていくことがわかる。また第1ピークの立ち上がり時間が早くなっている。
\begin{figure}[h]
	\centering
	\includegraphics[width=15cm]{figure/fig_3_2_10QW_ridge_L400_GS.png}
		\caption{10歪補償量子井戸リッジ導波路型レーザー $L$=400 \si{\micro\metre} の利得スイッチング光パルスの時間波形}
		\label{fig:fig_3_2_10QW_ridge_L400_GS}
\end{figure}

%横軸の時間同じにできないの?
図\ref{fig:fig_3_2_10QW_ridge_L500_GS}に共振器長$L$=500 \si{\micro\metre}の試料の時間波形を示す。(a)を見ると第1ピークが励起強度とともに増大していくことがわかる。(b)を見ると励起強度の増大とともに立ち上がり時間が早くなっている。さらに26.6 Vを超えると緩和振動の第2パルスが見られ始める。
\begin{figure}[h]
	\centering
	\includegraphics[width=15cm]{figure/fig_3_2_10QW_ridge_L500_GS.png}
		\caption{10周期歪補償量子井戸リッジ導波路型レーザー $L$=500 \si{\micro\metre} の利得スイッチング光パルスの時間波形}
		\label{fig:fig_3_2_10QW_ridge_L500_GS}
\end{figure}

\newpage
\subsubsection{10周期歪補償量子井戸リッジ導波路型レーザーの利得スイッチング実験のまとめ}
10周期歪補償量子井戸リッジ導波路型レーザーについてはどの全ての共振器長の試料についても典型的な利得スイッチング動作が見られた。


次図\ref{fig:fig_3_2_10QW_ridge_GS_FWHM}に10周期歪補償量子井戸リッジ導波路型レーザーの利得スイッチングパルスのパルス幅を示す。色分けは共振器長の違いを表す。
励起強度を増加していくとパルス幅は短くなる様子がどの共振器長でも見て取れる。$L$=300 \si{\micro\metre}では30 Vより強励起ではパルス幅は短くなることはなく29 ps程度で横ばいになっている。$L$=400 \si{\micro\metre}、$L$=500 \si{\micro\metre}でも印加電圧30 V付近でパルス幅の変化が小さくなっており、飽和が起きている。最短パルス幅は$L$=400 \si{\micro\metre}で26.5 psであった。共振器長依存性は見られなかった。
\begin{figure}[h]
	\centering
	\includegraphics[width=10cm]{figure/fig_3_2_10QW_ridge_GS_FWHM.png}
		\caption{10周期歪補償量子井戸リッジ導波路型レーザーの光パルス幅}
		\label{fig:fig_3_2_10QW_ridge_GS_FWHM}
\end{figure}

\subsection{利得スイッチング実験のまとめ}%===========================

3周期歪量子井戸リッジ導波路型レーザーと10周期歪補償量子井戸リッジ導波路型レーザーについての利得スイッチング実験の結果をまとめる。


\subsubsection{3周期歪量子井戸レーザーと10周期歪補償レーザーの結果の比較}
$L$=100 \si{\micro\metre}と200 \si{\micro\metre}については試料のマウントの方法に問題があり典型的な利得スイッチングパルスは得られなかった。共振器長が$L$=300 \si{\micro\metre}試料では最短パルス幅23.4 V印加で28.9 psであった。

10周期試料については典型的な利得スイッチングパルスが得られ、最短パルス幅は$L$=400 \si{\micro\metre}で26.5 psであった。

3周期と10周期試料ではどちらも30 ps程度に収束しておりモード利得の違いによるパルス幅の差異が明確ではないと考えられる。

パルス幅を制限する外的要因として駆動電気パルスの立ち上がり時間について詳細に調べることで解明が可能であると考えられる。
\subsubsection{利得スイッチングパルスの共振器長依存性について}
先行研究によると利得スイッチングパルスの立ち下がりは共振器寿命によって決定された。
そこで式(\ref{eq:tau_p})を用いて共振器寿命を計算してみると
\begin{table}[h]
  \caption{共振器寿命}
  \label{table:table_taup}
  \centering
  \begin{tabular}{lccc}
    \hline
    試料   &  内部損失$\alpha_{int} \rm{ /cm}$&共振器長  \si{\micro\metre} &共振器寿命 $\tau_{p}$  ps \\
    \hline \hline
     3周期歪量子井戸 &   11.8 &300 &2.3\\
    10周期歪補償量子井戸  & 18.0 &300&2.01\\
    \hline
  \end{tabular}
\end{table}

計算にはR=0.3, $n_{eq}$=3.5を用いた。

実験結果はパルス幅数十psであるので共振器寿命よりも1桁大きくなっている。共振器寿命のパルス幅の決定への寄与は小さいと考えられる。共振器長を変えたときにパルス幅が変わらないことの根拠となると考えられる。

		%実験結果
%\chapter{考察}
\label{chap:simulation}

\section{はじめに}

\section{パルス幅と共振器長の関係(10QW)}
\section{3QWと10QWの比較}
閾値の何倍で飽和しているため利得スイッチングの速さの限界が異なる?

実験での電圧から単位体積あたりの励起強度つまり励起キャリア密度を見積もった?
	%考察
% !TEX root = main.tex

\chapter{まとめと展望}
この章では本研究で得られた結論と展望について述べる。
\section{本研究のまとめ}%InGaAs系高利得半導体レーザーの開発および評価測定
\adsp02{本研究では応用上重要な電流注入型の1\si{ \micro\metre}波長帯InGaAs系半導体レーザーの利得スイッチング動作に着目した。}


利得スイッチングパルスの\adsp02{メカニズムはGaAs系材料を用いた光励起実験により解明が進められており、}\delsp02{短パルス化は}モード利得を大きくすることと共振器寿命を短くすることで\adsp02{短パルス化が}達成され得ることが示されていた。


本研究ではまずは\addspan{多重量子井戸化による}\delspan{利得層を厚く積むことにより}高利得化を意図した多重量子井戸レーザーを作製することを目的とした。
GaAsに対して格子定数が大きいInGaAs材料を厚く積層するためにバリア層に格子定数の小さいInGaPを用いた10周期歪補償量子井戸構造ウエハを作製した。また比較のために3周期歪量子井戸構造のウエハも作製した。\adsp02{それぞれの}エピウエハをデバイス化しブロードコンタクトレーザーとリッジ導波路型レーザーを作製した。

%マウントを行い電流注入実験を行った。

ブロードコンタクトレーザーに対して定常電流注入実験を行ったところ
%高い抵抗値と
電流が流れる幅を決める電極パッド幅に対して優位にが広がっていることがわかった。広がりは3周期歪量子井戸試料では60 \si{\micro\metre}程度
、10周期歪補償量子井戸試料では25$\sim$50 \si{\micro\metre}と見積もられた。閾値電流密度を見積もると3周期歪量子井戸レーザーでは0.20$\sim$0.35 $\rm{kA/cm^{2}}$、10周期歪補償量子井戸レーザーでは0.40 $\rm{kA/cm^2}$と見積もられた。また透明電流密度と微分利得係数の比はそれぞれ4.6倍、3.7倍と見積もられ、量子井戸の多重化によるモード利得の増大を確かめることができた。

リッジ導波路型レーザーについて定常電流注入実験を行ったところ。\delsp02{10周期試料に関して発振を確かめることができ、レーザーデバイスの品質が}10周期歪補償量子井戸レーザーに関して発光量の\delspan{減衰}\addspan{ドループ(入力増に対する出力低下)}が観測された。追加実験により原因が特定できると考えられる。


リッジ導波路型レーザーについて短パルス注入実験を行った。3周期歪量子井戸レーザーについては\delsp02{インピーダンス不整合のために}典型的な利得スイッチングパルスを観測することが困難であったが最短のパルス幅として28.9 psを得た。また10周期歪補償量子井戸レーザーに関しては典型的な利得スイッチングパルスが観測され、最短パルス幅は26.5psであった。\adsp02{先行研究では電流注入により5 sp程度のパルス幅の実現が報告されており、それには及ばなかった。しかし市販の半導体レーザーについて行った同様の実験では80から200 psのパルス幅を与え、本研究において開発したレーザーデバイスの短パルス発生における優位性を示した。}

利得スイッチングパルスのパルス幅に関してはモード利得あるいは共振器長の違いによるパルス幅の差異は明確ではなかった。この原因として電気信号の帯域による制限がかけられているものと考えられる。

\section{今後の展望}
\addspan{InGaPをバリア層に用いることで良質な10周期歪補償量子井戸レーザーを作製することができた。}
3周期歪量子井戸レーザーと10周期歪補償量子井戸レーザーで比較した\delspan{とき}\addspan{場合}モード利得の増大が見られ、エピウエハデザインの段階で期待した効果が見られた。\addspan{今後のレーザー開発においてさらに量子井戸数を増やすことでさらなる高利得化が見込める。}
リッジ導波路型レーザーの定常電流注入測定結果においてドループが見られたが、それを抑制するようなマウントを行い追加実験を行った。付録5.1節に示す。活性層部分の熱が逃げやすいようにエピダウンと呼ばれるマウント方法を用いたところドループが改善された。今後このようはマウントを主流に行うことで高密度励起や高出力化に適したデバイス作製ができるのではないかと考えられる。

キャリア広がり\addspan{が大きいという問題が見出された。}\delspan{も観測された}これについては部分的に追加実験を行った。それを付録5.2節に示す。リッジ導波路型レーザーにおいてリッジの両側に溝を形成し定常電流注入実験を行ったところ、閾値の低減が見られた。このことからウエハ内部でのキャリア広がりが示唆され、原因はInGaP層であると考えられる。今後の結晶成長においてはInGaP層の薄いデバイスを作製することがより高品質なレーザーを開発することができる。


電流注入利得スイッチング実験についてはパルス幅は電気パルスの帯域制限を受けていると考えられるため\addspan{高周波実装と}駆動系の改善を行いたい。十分短い電気パルスで強く励起した場合にこそ高利得化の利点が見られ\addspan{る}と考えられる。	%まとめと展望

\chapter{付録}

%\section{deconvolutionの計算}
%\section{駆動系の電気信号}
%\section{GaAsの発振}
%線形離島の領域なの?
\section{リッジ導波路型レーザーにおいて観測されたドループに関する追加実験}

本節ではリッジ導波路型レーザーのILカーブ測定で観測されたドループを軽減する方法を模索するために行った実験について述べる。ドループは試料の温度上昇によるものだと考え、活性層部分の熱が逃げやすいよう上下をひっくり返してサブマントにダイボンディングを行った。このマウント方法をエピダウンと呼ぶ。その写真を図\ref{fig:fig_5_1_epidown_mount}に示す。写真は3周期歪量子井戸レーザー$L=300\ \si{\micro\metre}$である。裏向きの試料がALNサブマウントに大ボンディングされている。図では試料のn側コンタクトが見えており、そこから出たワイヤーがプローバーでさわるためのパターンに配線されている。このマウント方法は活性層がサブマウント、サブマウントが乗っている銅板に近いため熱が逃げやすい。
\begin{figure}[h]
	\includegraphics[width=15cm]{figure/fig_5_1_epidown_mount.png}
	\caption{エピダウン試料マウントの様子}
	\label{fig:fig_5_1_epidown_mount}
\end{figure}


図\ref{fig:fig_5_1_epidown_IL}にエピダウン試料の定常電流注入実験結果のILカーブを示す。測定は全て2\ $\si{\micro s}$パルスを2\ $\si{m s}$周期で印加しており(デュティー比1:1000)、プロットは片側端面からの発光強度を1000倍した値である。試料は全て3周期歪量子井戸リッジ導波路型レーザーの結果である。

まず、破線はエピダウンを行っていない3周期試料($L=500\ \si{\micro\metre}$)のILカーブであり、色分けは同じ試料に対する測定回数を表す。1回目の測定では200\ \si{mA}程度まで電流を流した結果、ドループは見られなかった。2回目の測定では300\ \si{mA}まで電流を流すとドループが観測された。さらに測定を重ねて行うと発光量はさらに少なくなっていく様子が見られた。

次に実線及び点線のプロットはエピダウンを行った試料のILカーブを示す。色分けはそれぞれ共振器長を表している。これらの試料では$L=300, 400 \ \si{\micro\metre}$では電流400\ mAで発光強度が60\ mWとエピダウンしていない試料よりも高い値を持っている。また、$L=1000\ \si{\micro\metre}$においては電流1000\ mAまで流してもドループは見られず80\ mW程度までの発光が確認された。

この実験からエピダウンにより活性層の温度上昇が抑制されまたドループも抑制されることがわかった。
\begin{figure}[h]
	\centering
	\includegraphics[width=10cm]{figure/fig_5_1_epidown_IL.png}
	\caption{エピダウン試料のILカーブ}
	\label{fig:fig_5_1_epidown_IL}
\end{figure}

%\clearpage
\section{リッジ導波路型レーザーにおけるFIB加工レーザー試料の測定}
本節ではFIB加工(集束イオンビーム、Focused Ion Beam)を用いた試料作製とその結果について示す。
ブロードコンタクトレーザー試料の測定結果からキャリアがパッド幅方向(リッジ導波路型レーザーにおいてはリッジ幅方向)にキャリアが広がっている可能性が示唆されていた。そこでFIB加工によりリッジの両脇に活性層よりも十分深い溝を形成しキャリアの拡散が起こらないような試料を作製し、定常電流注入実験を行った。実験には3周期歪量子井戸リッジ導波路型レーザー($L=300\ \si{\micro\metre}$)を用いた。またFIB加工はNTT-AT社に外注した。

図\ref{fig:fig_5_2_FIB_facet}にFIB加工を行った試料の端面方向の写真を示す。リッジの両側に深い溝が形成されていることがわかる。
溝の幅は$6\sim17\ \si{\micro\metre}$の試料を作製した。
\begin{figure}[h]
	\centering
	\includegraphics[width=8cm]{figure/fig_5_2_FIB_facet.png}
	\caption{FIB試料の端面写真}
	\label{fig:fig_5_2_FIB_facet}
\end{figure}


図\ref{fig:fig_5_2_FIB_IL}にFIB加工試料の定常電流注入測定結果のILカーブを示す。色分けはレーザーバー素子の番号を表し、それぞれFIB加工の溝の間隔が紫:17\ $\si{\micro\metre}$、青:16\ $\si{\micro\metre}$、緑:12\ $\si{\micro\metre}$、橙:10\ $\si{\micro\metre}$、赤: 6\ $\si{\micro\metre}$と異なる。それぞれの色について点線と実線があるが、実線は950$\ \si{\micro\metre}$ロングパルフィルタを入れて測定を行った結果、点線はフィルタを入れずに行った結果を表す。フィルタを入れた際にはInGaAs活性層の発光のみを検出し、フィルタを外した際にはGaAsの発光も検出されているため点線の方が大きい値となる部分が生じてしまっている。

これを見ると閾値電流は最小で60 mA程度とFIB加工を行っていない試料の結果(図\ref{fig:fig_3_2_3QW_ridge_IL}及び図\ref{fig:fig_3_2_3QW_ridge_Ith}では最小80 mA)と比較すると小さくなっている。FIB加工による電流の流れる幅を制限したことにより閾値低減が行われたことがわかる。このことから電流が広がって流れているのではないかと考察される。
\begin{figure}[h]
	\centering
	\includegraphics[width=10cm]{figure/fig_5_2_FIB_IL.png}
	\caption{FIB試料のILカーブ}
	\label{fig:fig_5_2_FIB_IL}
\end{figure}



\clearpage
\section{格子定数、$E_{g}$の計算}
$\rm{In_{1-x}Ga_{x}As_{y}P_{1-x}}$の格子状数は\cite{ref_iga}より
\begin{eqnarray}
a=5.8687-0.4176x+0.1896y+0.0125xy\\
\end{eqnarray}
と計算した。$\rm{Al_{1-x}Ga{x}As}$の格子定数は
\begin{eqnarray}
a=5.65325x+5.6605(1-x)
\end{eqnarray}
として計算した。
$E_{g}$は下の表の式を用いて計算を行った。
\begin{table}[h]
  \caption{3周期歪量子井戸ブロードコンタクトレーザーの電流広がり}
  \label{table:table_Eg}
  \centering
  \begin{tabular}{cc}
    \hline
    材料& $E_{g}$の式   \\
    \hline \hline
     $\rm{Ga_{x}In_{1-x}As}$ &$0.324+0.7x+0.4x^2$   \\
    $\rm{Al_{x}Ga_{1-x}As}$& $1.420+1.087x+0.438x^2$ \\
    $\rm{Ga_{x}In_{1-x}P}$& $1.351+0.643x+0.786x^2$ \\ 
    $\rm{GaAs_{x}P_{1-x}}$&$2.750-1.502x+0.176x^2$\\
    \hline
  \end{tabular}
\end{table}
%\subsection{InGaPのトンネル}

 		%付録
%% reference
\begin{thebibliography}{99}
\bibitem{ref_1_1_GS} Shaoqiang Chen, Masahiro Yoshita, Takashi Ito, Toshimitsu Mochizuki, Hidefumi Akiyama, Hiroyuki Yokoyama, Kenji Kamide, and Tetsuo Ogawa. Analysis of Gain-Switching Characteristics Including Strong Gain Saturation Effects in Low-Dimensional Semiconductor Lasers. \sl Japanese Journal of Applied  Physics \rm, Vol 5,098001,2012.
\bibitem{ref_iga} 伊賀健一 (1994)『半導体レーザ』, オーム社
\bibitem{ref_t_ito}Takashi Ito
, Hidekazu Nakamae, Yuji Hazama, Takahiro Nakamura, Shaoqiang Chen, Masahiro Yoshita, Changsu Kim, Yohei Kobayashi and Hidefumi Akiyama.
Femtosecond pulse generation beyond photon lifetime limit in gain-switched semiconductor lasers .\sl Communications Physics\rm , DOI: 10.1038/s42005-018-0045-0, 2018. 

\end{thebibliography}
 %参考文献

謝辞\\
%\begin{comment}
秋山英文教授には2年間の修士課程を通して物理への向き合い方だけではなく、人としての生き方を教えていただいたように思う。先生の下で過ごすことができたことを誇りに思う。


挟間優治助教授は居室での席が近かったことも合間って研究室生活を意義のあるものにしていただいたと感じている。物理に関する疑問からプログラムの書き方などなど多岐にわたるドバイスをいただいた。感謝申し上げる。


秋山研究室メンバーの同期である柴田桂成氏は本質的な物理への探究心を持った人物であり、日々の会話の中で多大なる刺激をもらうことができた。


先輩である中村考宏氏からはもっとも多くのことを学ばせていただいた。実験具体的な手法から半導体プロセスまでありとあらゆる場面で学ばせていただいた。中前秀一氏にはレーザー動作に関する基礎物理の議論をしていただいたことが記憶に新しい。わからないことがあればいつでも議論をぶつけられるウェルカムな先輩であった。修論のアドバイス本当にありがとうございました。


特任研究員の金昌秀氏には劈開、電極の蒸着やエッチングプロセスを共ににやっていただいたことが学びにつながった。非常に丁寧な作業姿からは物事に取り組む姿勢を感じた。

陶仁春氏はときを同じくして秋山研究室に所属したメンバーの一人であり、日常生活から利得スイッチング実験まで共に学ぶことが多かった。陶氏の研究への真摯な態度と日常生活での気の利いた冗談はとても心地の良いものであった。


薄倉淳子氏には日常会話で気分を明るくしていただいた。すばらしい画像解析プログラムを作られており、私がpythonをかじるきっかけとなった。興味の幅を広げてくれたことに感謝申し上げる。
廣瀬修平氏は数少ない後輩であり日々のたわいもない雑談がどれほど励みになったであろうか。全く目的もなく話にいく私に付き合っていただいたこと感謝を述べる。

その他秋山研究室で共に過ごしたOB、OGのみなさまには感謝を申し上げる。
奥哲氏はレーザーデバイス開発の中枢を担っていただいた。溢れ出る豊富な経験ゆえに、プロセスの作業を一緒に行うだけでも学ぶところが多かった。素人の質問にも懇切丁寧に説明をしていただき大変理解が深まったことは言うまでもない。


また試料提供をしていただいたNTT-AT社の職員の方々、およびオプトウェル社の職員の方々にも感謝の意を示したい。


最後にときおり進路や学業の相談に乗ってくれた父と多忙の中実家で日々の生活を支えてくれた母に感謝の意を表して締めとしたい。

%s先生、NTT-AT奥哲様 
%\end{comment}
\end{document}
