
\chapter{付録}

%\section{deconvolutionの計算}
%\section{駆動系の電気信号}
%\section{GaAsの発振}
%線形離島の領域なの?
\section{格子定数、$E_{g}$の計算}
$\rm{In_{1-x}Ga_{x}As_{y}P_{1-x}}$の格子状数は\cite{ref_iga}より
\begin{eqnarray}
a=5.8687-0.4176x+0.1896y+0.0125xy\\
\end{eqnarray}
と計算した。$\rm{Al_{1-x}Ga{x}As}$の格子定数は
\begin{eqnarray}
a=5.65325x+5.6605(1-x)
\end{eqnarray}
として計算した。
$E_{g}$は下の表の式を用いて計算を行った。
\begin{table}[h]
  \caption{3周期歪量子井戸ブロードコンタクトレーザーの電流広がり}
  \label{table:table_Eg}
  \centering
  \begin{tabular}{cc}
    \hline
    材料& $E_{g}$の式   \\
    \hline \hline
     $\rm{Ga_{x}In_{1-x}As}$ &$0.324+0.7x+0.4x^2$   \\
    $\rm{Al_{x}Ga_{1-x}As}$& $1.420+1.087x+0.438x^2$ \\
    $\rm{Ga_{x}In_{1-x}P}$& $1.351+0.643x+0.786x^2$ \\ 
    $\rm{GaAs_{x}P_{1-x}}$&$2.750-1.502x+0.176x^2$\\
    \hline
  \end{tabular}
\end{table}
%\subsection{InGaPのトンネル}

