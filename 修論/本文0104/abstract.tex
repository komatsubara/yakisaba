\begin{abstract}
 半導体レーザーは世の中で広く用いられている発光デバイスである。小型、高安定などの利点を有し非常に扱いが容易である。そのような半導体デバイスから直接ps程度の超短パルスを発生させることは、超短パルスを用いた様々な基礎研究の成果を技術的に応用する足がかりとなり得る。
 
 半導体から超短パルス光を得る手法として利得スイッチングを取り上げる。
 利得スイッチングはns程度の短い電気パルスを注入することそれより短い数十psの光パルスを得る技術である。電流による直接変調のみで実現可能でありまた複雑な構造を必要としないことから簡便に実現が可能な手法といえる。
 %電流注入により半導体内部の利得を増幅し、反転分布を作る。
 
 利得スイッチングパルスによるパルス発生に関しては古くから研究がなされてきているが、近年の研究で光パルスの立ち上がりは半導体材料の利得飽和が、立ち下がりは共振器寿命が決めているということが報告されている。共振器寿命は半導体レーザーの共振器長を短くすることで短パルス化が見込まれる。一方利得飽和に関してはレーザーデバイス設計の時点では決めることができない。そこで利得飽和の代わりに光閉じ込め係数を高くすることで高い利得を実現しパルスの短パルス化を図れることが示唆されている。量子井戸レーザーにおいて閉じ込め係数は量子井戸の数を増やすことで実現される。
 

 そこで本研究では10周期多重量子井戸半導体レーザーをデザインおよび開発を行った。同時に3周期多重量子井戸レーザーも開発し、電流注入実験を行い比較したところ高い利得であることが示された。
 
 また利得スイッチング動作を試みたところ10周期多重量子井戸レーザーでは最短で26.5 psというパルス幅を与えた。
 
 %共振器長あるいは量子井戸の層数を変えたことによる短パルス化の効果はあまり見られなかった。
\end{abstract}