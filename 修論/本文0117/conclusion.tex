% !TEX root = main.tex

\chapter{まとめと展望}
この章では本研究で得られた結論と展望について述べる。
\section{本研究のまとめ}%InGaAs系高利得半導体レーザーの開発および評価測定
\adsp02{本研究では応用上重要な電流注入型の1\si{ \micro\metre}波長帯InGaAs系半導体レーザーの利得スイッチング動作に着目した。}


利得スイッチングパルスの\adsp02{メカニズムはGaAs系材料を用いた光励起実験により解明が進められており、}\delsp02{短パルス化は}モード利得を大きくすることと共振器寿命を短くすることで\adsp02{短パルス化が}達成され得ることが示されていた。


本研究ではまずは\addspan{多重量子井戸化による}\delspan{利得層を厚く積むことにより}高利得化を意図した多重量子井戸レーザーを作製することを目的とした。
GaAsに対して格子定数が大きいInGaAs材料を厚く積層するためにバリア層に格子定数の小さいInGaPを用いた10周期歪補償量子井戸構造ウエハを作製した。また比較のために3周期歪量子井戸構造のウエハも作製した。\adsp02{それぞれの}エピウエハをデバイス化しブロードコンタクトレーザーとリッジ導波路型レーザーを作製した。

%マウントを行い電流注入実験を行った。

ブロードコンタクトレーザーに対して定常電流注入実験を行ったところ
%高い抵抗値と
電流が流れる幅を決める電極パッド幅に対して優位にが広がっていることがわかった。広がりは3周期歪量子井戸試料では60 \si{\micro\metre}程度
、10周期歪補償量子井戸試料では25$\sim$50 \si{\micro\metre}と見積もられた。閾値電流密度を見積もると3周期歪量子井戸レーザーでは0.20$\sim$0.35 $\rm{kA/cm^{2}}$、10周期歪補償量子井戸レーザーでは0.40 $\rm{kA/cm^2}$と見積もられた。また透明電流密度と微分利得係数の比はそれぞれ4.6倍、3.7倍と見積もられ、量子井戸の多重化によるモード利得の増大を確かめることができた。

リッジ導波路型レーザーについて定常電流注入実験を行ったところ。\delsp02{10周期試料に関して発振を確かめることができ、レーザーデバイスの品質が}10周期歪補償量子井戸レーザーに関して発光量の\delspan{減衰}\addspan{ドループ(入力増に対する出力低下)}が観測された。追加実験により原因が特定できると考えられる。


リッジ導波路型レーザーについて短パルス注入実験を行った。3周期歪量子井戸レーザーについては\delsp02{インピーダンス不整合のために}典型的な利得スイッチングパルスを観測することが困難であったが最短のパルス幅として28.9 psを得た。また10周期歪補償量子井戸レーザーに関しては典型的な利得スイッチングパルスが観測され、最短パルス幅は26.5psであった。\adsp02{先行研究では電流注入により5 sp程度のパルス幅の実現が報告されており、それには及ばなかった。しかし市販の半導体レーザーについて行った同様の実験では80から200 psのパルス幅を与え、本研究において開発したレーザーデバイスの短パルス発生における優位性を示した。}

利得スイッチングパルスのパルス幅に関してはモード利得あるいは共振器長の違いによるパルス幅の差異は明確ではなかった。この原因として電気信号の帯域による制限がかけられているものと考えられる。

\section{今後の展望}
\addspan{InGaPをバリア層に用いることで良質な10周期歪補償量子井戸レーザーを作製することができた。}
3周期歪量子井戸レーザーと10周期歪補償量子井戸レーザーで比較した\delspan{とき}\addspan{場合}モード利得の増大が見られ、エピウエハデザインの段階で期待した効果が見られた。\addspan{今後のレーザー開発においてさらに量子井戸数を増やすことでさらなる高利得化が見込める。}
リッジ導波路型レーザーの定常電流注入測定結果においてドループが見られたが、それを抑制するようなマウントを行い追加実験を行った。付録5.1節に示す。活性層部分の熱が逃げやすいようにエピダウンと呼ばれるマウント方法を用いたところドループが改善された。今後このようはマウントを主流に行うことで高密度励起や高出力化に適したデバイス作製ができるのではないかと考えられる。

キャリア広がり\addspan{が大きいという問題が見出された。}\delspan{も観測された}これについては部分的に追加実験を行った。それを付録5.2節に示す。リッジ導波路型レーザーにおいてリッジの両側に溝を形成し定常電流注入実験を行ったところ、閾値の低減が見られた。このことからウエハ内部でのキャリア広がりが示唆され、原因はInGaP層であると考えられる。今後の結晶成長においてはInGaP層の薄いデバイスを作製することがより高品質なレーザーを開発することができる。


電流注入利得スイッチング実験についてはパルス幅は電気パルスの帯域制限を受けていると考えられるため\addspan{高周波実装と}駆動系の改善を行いたい。十分短い電気パルスで強く励起した場合にこそ高利得化の利点が見られ\addspan{る}と考えられる。