% !TEX root = main.tex
\begin{abstract}
  半導体レーザーは世の中で広く用いられている発光デバイスである。小型、高安定などの利点を有し非常に扱いが容易である。そのような半導体デバイスから直接ps程度の超短パルスを発生させることは、超短パルスを用いた様々な基礎研究の成果を技術的に応用する足がかりとなり得る。
 
% \addspan{本研究では応用上重要な電流注入型の1\si{ \micro\metre}波長帯InGaAs系半導体レーザーの高速化を目的として研究を行った。}
 %\addspan{1 \si{\micro\metre}波長帯InGaAs系半導体レーザーは、先行研究開発例が少ないため、設計・試作・評価計測を半導体の基礎物理に立ち戻って進める必要がある。}
 
 半導体から超短パルス光を得る手法として利得スイッチングを取り上げる。
 利得スイッチングはns程度の短い電気パルスを注入すること\addspan{で}それより短い数十ps\addspan{程度}の光パルスを得る技術である。電流による直接変調のみで実現可能であり\delspan{また}\addspan{、}複雑な構造を必要と\addspan{しない。\delsp02{したがって}基本的な半導体内部のキャリアダイナミクスを直接反映する物理現象を理解することに繋がる。}\delspan{しないことから簡便に実現が可能な手法といえる。}
 %電流注入により半導体内部の利得を増幅し、反転分布を作る。
 
 利得スイッチング\delspan{パルスによる}\addspan{法を用いた}パルス発生に関しては古くから研究がなされてきているが、近年の研究\delspan{で}\addspan{において}光パルスの立ち上がりは半導体材料の\delspan{利得飽和}\addspan{飽和モード利得}が、立ち下がりは共振器寿命が決めているということが報告されている。共振器寿命は半導体レーザーの共振器長を短くすることで
 %\addspan{短くすることができ、
 \addspan{光パルスの}短パルス化が見込まれる。一方\addspan{飽和モード利得は}\delspan{利得飽和に関してはレーザーデバイス設計の時点では決めることができない。そこで利得飽和の代わりに}光閉じ込め係数を高くすることで高い利得を実現し\addspan{光}パルスの短パルス化を図れることが\addspan{GaAs系半導体レーザーの光励起実験などにより}示唆されている。
 
  \addspan{本研究では応用上重要な電流注入型の1\si{ \micro\metre}波長帯InGaAs系半導体レーザーの高速化を目的として研究を行った。}
 \addspan{利得スイッチング用1 \si{\micro\metre}波長帯InGaAs系半導体レーザーは、先行研究\adsp02{や}開発例が少ないため、設計・試作・評価計測を半導体の基礎物理に立ち戻って進める必要がある。}

\addspan{本研究では}量子井戸レーザーにおいて光閉じ込め係数の増大は量子井戸の数を増やすことで実現される\addspan{ことに着目し、}\delspan{。}
\delspan{そこで本研究では10周期}多重量子井戸半導体レーザー\delspan{を}\addspan{の}デザインおよび開発を行った。\delspan{同時に3周期多重量子井戸レーザーも開発し}\delspan{、}%\addspan{た。}
\addspan{InGaAs系材料はGaAs基板に対する格子定数の違いからそのままでは量子井戸層の層数を大きくすることはできない。\adsp02{また、格子欠陥が生じるなど材料の品質低下をまねきやすい。}そこで格子定数の小さい\delsp02{InGaP}\adsp02{GaAsP}をバリア層に用いることで10周期歪補償量子井戸レーザーを作製した。\adsp02{電流注入では光励起と異なり、電子と正孔の密度を等しく注入できるか否かは自明ではない。よって多重量子井戸の層厚を増やすことで単純に高利得化を実現できるとは限らない。比較のために}\delsp02{同時に}歪補償を行わずに結晶成長させた3周期歪量子井戸レーザーの試作も行った。}

\addspan{作製したレーザーデバイスに対して}定常電流注入実験を行い\addspan{閾値電流及びスロープ効率の見積もりを行った。これらの値を解析することでモード利得が算出できる。10周期と3周期の2種類の井戸数のレーザーデバイスについて}比較したところ\addspan{、量子井戸の多重化によるモード利得の増大が確かめられた。}\delspan{高利得であることが示された。}
 
 また利得スイッチング動作を試みたところ10周期多重量子井戸レーザーでは最短で26.5 psというパルス幅を与えた。\adsp02{先行研究では電流注入による5 ps程度の報告がなされており、それには及ばなかった。しかし市販の普及品で同様の測定をしたところ80から200 psのパルス幅を与え、本研究において開発したレーザーデバイスの短パルス発生における優位性を示した。}
 %共振器長あるいは量子井戸の層数を変えたことによる短パルス化の効果はあまり見られなかった。
\end{abstract}