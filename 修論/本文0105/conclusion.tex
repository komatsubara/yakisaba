% !TEX root = main.tex

\chapter{まとめと展望}
この章では本研究で得られた結論と展望について述べる。
\section{本研究のまとめ}%InGaAs系高利得半導体レーザーの開発および評価測定
利得スイッチングパルスの短パルス化はモード利得を大きくすることと共振器寿命を短くすることで達成され得ることが示されていた。本研究では利得層を厚く積むことにより高利得化を意図した多重量子井戸レーザー作製すること、また作製したデバイスの特性を測定することを目的とした。レーザー試料作製を行い定常電流および短パルス電流を注入する実験を行った。

%1\si{\micro\metre}帯の
GaAsに対して格子定数が大きいInGaAs材料を厚く積層するためにバリア層に格子定数の小さいInGaPを用いた10周期歪補償量子井戸構造ウエハを作製した。また比較のために3周期歪量子井戸構造のウエハも作製した。エピウエハをデバイス化しブロードコンタクトレーザーとリッジ導波路型レーザーを作製した。マウントを行い電流注入実験を行った。

ブロードコンタクトレーザーに対して定常電流注入実験を行ったところ
%高い抵抗値と
電流が流れる幅を決める電極パッド幅に対して優位にが広がっていることがわかった。広がりは3周期歪量子井戸試料では60 \si{\micro\metre}程度
、10周期歪補償量子井戸試料では25$\sim$50 \si{\micro\metre}と見積もられた。閾値電流密度を見積もると3周期歪量子井戸レーザーでは0.20$\sim$0.35 $\rm{kA/cm^{2}}$、10周期歪補償量子井戸レーザーでは0.40 $\rm{kA/cm^2}$と見積もられた。また透明電流密度と微分利得係数の比はそれぞれ4.6倍、3.7倍と見積もられ多重量子井戸化の効果をみることができた。

リッジ導波路型レーザーについて定常電流注入実験を行ったところ。10周期歪補償量子井戸レーザーに関して発光量の減衰が観測された。追加実験により原因が特定できると考えられる。


リッジ導波路型レーザーについて短パルス注入実験を行った。3周期歪量子井戸レーザーについてはインピーダンス不整合のために典型的な利得スイッチングパルスをみることが困難であったが最短のパルス幅として28.9 psを得た。また10周期歪補償量子井戸レーザーに関しては典型的な利得スイッチングパルスを観測することができ、最短パルス幅は26.5psであった。

モード利得あるいは共振器長の違いによるパルス幅の差異は明確ではなかった。この原因として電気信号の帯域による制限がかけられているものと考えられる。

\section{今後の展望}
3周期歪量子井戸レーザーと10周期歪補償量子井戸レーザーで比較したときモード利得の増大が見られた。エピウエハデザインの段階で期待した効果が見えた。一方でInGaP層によるキャリア広がりも観測された。このことからInGaP層の薄い構造のデバイスを作製することがより高品質なレーザーを開発する上で肝心となるのではないかという知見を得た。


電流注入利得スイッチング実験についてはパルス幅は電気パルスの帯域制限を受けていると考えられるため駆動系の改善を行いたい。十分短い電気パルスで強く励起した場合にこそ高利得化の利点が見られと考えられる。