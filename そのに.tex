

\documentclass{jsarticle}
\usepackage{mathrsfs}
\usepackage{comment}
\usepackage{amsmath}
%\makeatletter
   % \renewcommand{\theequation}{
    %\thesection.\arabic{equation}}
    %\@addtoreset{equation}{section}
\makeatother
\usepackage[dvipdfmx]{graphicx}
\usepackage{bm}


%プリアンブル
\title{物質科学概論II}
\author{35176043 小松原望}
%\date{2017/07/25}

\begin{document}
%    \maketitle % タイトルを出力
%\begin{figure}[htbp]
 %\begin{center}
  %\includegraphics [width=80mm]{test2.pdf}
 %\end{center}
 %\caption{FeCo状態密度図}
 %\label{fig:one}
%\end{figure}

物質科学概論II\\
\quad35176043 \quad 小松原望\\
1.(1)磁気モーメント$\mu$がボルツマン分布に従っているとすると全体の磁化は

\begin{eqnarray}
M=N\mu\int_0^\pi \cos\theta P(\theta)d\theta\\
\Big(P(\theta)d\theta=\frac{\exp(\frac{\mu H\cos\theta}{kT})2\pi\sin\theta d\theta}{\int_0^\pi \exp(\frac{\mu H\cos\theta}{kT})2\pi\sin\theta d\theta}\Big)\nonumber\\
=N\mu\frac{\int_0^\pi\exp(\frac{\mu H\cos\theta}{kT})2\pi\sin\theta \cos\theta d\theta}{\int_0^\pi \exp(\frac{\mu H\cos\theta}{kT})2\pi\sin\theta d\theta}
\end{eqnarray}
$\alpha=\mu H/kT$,$x=\cos\theta$とおくと$\-sin\theta d\theta=dx$となるから\\
\begin{eqnarray}
M=N\mu\frac{\int_{-1}^1\mathrm{e}^{\alpha x}xdx}{\int_{-1}^1\mathrm{e}^{\alpha x} dx}
\end{eqnarray}
ここで
\begin{eqnarray}
\int_{-1}^1\mathrm{e}^{\alpha x} dx=\frac{1}{\alpha}[\mathrm{e}^{\alpha x}]_{-1}^1=\frac{\mathrm{e}^\alpha - \mathrm{e}^{-\alpha}}{\alpha} \nonumber\\
\int_{-1}^1\mathrm{e}^{\alpha x}xdx=[\frac{1}{\alpha}\mathrm{e}^{\alpha x}x]_{-1}^1-\frac{1}{\alpha}\int_{-1}^1\mathrm{e}^{\alpha x}dx \nonumber\\
=\frac{\mathrm{e}^\alpha+\mathrm{e}^{-\alpha}}{\alpha}-\frac{\mathrm{e}^\alpha - \mathrm{e}^{-\alpha}}{\alpha^2}\nonumber
\end{eqnarray}
より
\begin{eqnarray}
M&=&N\mu\Big(\frac{\mathrm{e}^\alpha+\mathrm{e}^{-\alpha}}{\mathrm{e}^{\alpha}-\mathrm{e}^{-\alpha}}-\frac{1}{\alpha}\Big)\\
&=&N\mu(\coth \alpha -\frac{1}{\alpha})\nonumber\\
&=&N\mu L(\alpha)
\end{eqnarray}
(Lはランジュバン関数)\quad$\alpha<<1\quad(T>>1)$の場合は
\begin{eqnarray}
\L(\alpha)=\frac{\alpha}{3}-\frac{\alpha^3}{45}-\cdot\cdot\cdot
\end{eqnarray}
第一項のみを取り出すと
\begin{eqnarray}
M=N\mu\frac{\alpha}{3}=\frac{N\mu^2H}{3kT}=\frac{CH}{T}\\
(C\equiv\frac{N\mu^2}{3k})
\end{eqnarray}
磁化率は$\chi=M/H$より
\begin{eqnarray}
\chi=\frac{C}{T}
\end{eqnarray}
\\\quad(2)$H\rightarrow H+H_{eff}=H+\lambda M$の置き換えをすると
\begin{eqnarray}
\alpha=\frac{\mu (H+\lambda M)}{kT} 
\end{eqnarray}
と置き換えられる。$\alpha<<1\quad(T>>1),L(\alpha)\approx\alpha/3$とすると全磁化Mは
\begin{eqnarray}
M=N\mu\frac{\alpha}{3}&=&\frac{N\mu^2(H+\lambda M)}{kT}\\
&=&\frac{C}{T}(H+\lambda M)\nonumber\\
\Big( C&\equiv&\frac{N\mu^2}{3k}\Big)\nonumber\\
MT&=&C(H+\lambda M)\nonumber\\
(T-C\lambda)M&=&CH\nonumber\\
M&=&\frac{CH}{T-C\lambda}
\end{eqnarray}
磁化率は
\begin{eqnarray}
\chi=\frac{M}{H}=\frac{C}{T-C\lambda}=\frac{C}{T-\Theta} \\
(\Theta\equiv C\lambda )
\end{eqnarray}
自発磁化$\lambda M$を持つ強磁性体では常磁性体の性質(式(9))が$T>>\Theta$の領域で成り立つことを示している。しかし$\Theta$近傍では法則からのずれが大きくなる。
%また常磁性体が
\\
\\2.(1)結晶内電子の運動方程式。Drude Modelでは電子は速度に比例した摩擦力を受けると考えて,電子の移動距離を$x$とすると
\begin{equation}
m^\ast\frac{d^2\bm{x}}{dt^2}+\frac{m^\ast}{\tau}\frac{d\bm{x}}{dt}=-e\bm{E}
\end{equation}
と書ける。$\tau$は緩和時間$(\tau\propto 1/\alpha)$。$m^\ast$は有効質量。右辺は電場から受ける力を表す。(電子が原子核に束縛されているとしたローレンツモデルにおいては
\begin{equation}
m^\ast\frac{d^2\bm{x}}{dt^2}+\frac{m^\ast}{\tau}\frac{d\bm{x}}{dt}+m^\ast\omega_0\omega^2=-e\bm{E}
\end{equation}
とポテンシャルに束縛される項が加えられる。$\omega$は$\bm{x}$が振動するとした時の振動周波数、$\omega0$は物質の固有振動数を表す。)\\
\quad 磁場によって駆動される磁壁の運動方程式はマクロには、移動距離を$x$として
\begin{eqnarray}
m\frac{d^2\bm{x}}{dt^2}+\alpha\frac{d\bm{x}}{dt}+A \bm{x}=2\bm{M}_s\bm{H}
\end{eqnarray}
と書ける。右辺は磁壁を動かそうと磁場から受ける圧力を表し、磁壁の移動によるエネルギー変化が単位面積当たり$-2M_sH\Delta x$で与えられることによる。$A$は磁壁のポテンシャルエネルギーの変化の曲率を表す。$\alpha$はダンピング定数。$m$は見かけの質量。\\
\quad ドルーデモデル(ローレンツモデル)の運動方程式と磁壁の運動方程式を見比べると、どちらも$x$の二階微分の項、ダンピングを表す項、ポテンシャル束縛の項、外部の場から受ける力の項から成り立っていることがわかる。\\
(2)ドルーでモデルにおいて、静電場が一様に印加されており電子の速度が一定になっているとすると、式(15)において$d^2\bm{x}/dt^2=0$として移動度は
\begin{eqnarray}
\frac{d\bm{x}}{dt}=-\frac{\tau e \bm{E}}{m^\ast}
\end{eqnarray}
となる。\\
\quad 磁壁を横から見た模式図を図1に示す。左側は+z方向、右側はーz方向に磁化しているとする。磁璧の厚さは$\delta$である。
\begin{figure}[htbp]
\begin{center}
 \includegraphics [width=80mm]{wall.pdf}
 \end{center}
 \caption{磁璧を横から見た図}
 \label{fig:one}
\end{figure}
yの位置では磁気モーメントはz方向に対し$\theta=y(\pi/\delta)$だけ傾いている。外部からの磁界$H_y$が生じるとスピンは回転し、その速度は
\begin{equation}
\frac{d\theta}{dt}=\gamma H_y
\end{equation}
である。$\gamma$は磁気回転比。磁璧のy方向への移動速度は
\begin{equation}
v=\frac{dy}{dt}=\frac{\delta}{\pi}\frac{d\theta}{dt}=\frac{\delta\gamma H_y}{\pi}
\end{equation}
となる。また、磁璧の運動エネルギーと地場からのエネルギーが等しいと置くと
\begin{equation}
-\frac{1}{2}H_yI_y\delta=\frac{1}{2}mv^2
\end{equation}
となり、見かけの質量は
\begin{eqnarray}
m=\frac{H_yI_y\delta}{v^2}=-\frac{H_yI_y\delta\pi^2}{\delta^2\gamma^2H_y^2}=\frac{\mu_0\pi^2}{\gamma^2\delta}
\end{eqnarray}
で与えられる。mが$1/\delta$に比例する結果が得られる。電子と磁壁を比較すると、どちらも速度は質量に反比例し、外部からの場に比例するという特徴を持つ。制動についてもどちらも物質内部の不均一性によって決まると考えられる。これは例えば結晶中に入ってしまった不純物や格子欠陥などが挙げられる。\\
3.(1)電気抵抗の外部磁場依存性の図を図2に示す。
\begin{figure}[htbp]
\begin{center}
 \includegraphics [width=80mm]{R.pdf}
 \end{center}
 \caption{電気抵抗の磁場依存性}
 \label{fig:one}
\end{figure}
まず正常磁気抵抗効果により磁気抵抗が印加磁場の二乗に比例して大きくなる(磁場が小さいとき)。これは伝導キャリアがローレンツ力をうけて物質中を移動する距離が長くなることに起因する。この場合の抵抗の変化は物質の結晶方位・磁場方向・電流方向の関係によって変わる。\\
\quad 一方強磁性金属ではローレンツ力以外に異常磁気抵抗効果と呼ばれる抵抗効果がある。磁化が飽和するまでの異方性磁気抵抗効果(AMR)と飽和した後の強制効果がある。磁化と電流が平行である場合よりも垂直である場合の方が抵抗は小さい。図には垂直な場合を示した。AMRはスピンー軌道相互作用により、磁化の向きに依存してd電子の波動巻子の広がりが変化するため電子の散乱確率が磁化と電流の方向に依存することによる。強磁性体では有限温度ではスピン無秩序が高く、この無秩序スピンが散乱ポテンシャルとして作用する。外部磁場が大きくなると秩序度が進行し、散乱が小さくなり抵抗も小さくなる。
\quad 磁化が飽和した後はAMRも飽和する。しかし抵抗はわずかに減少しこれを強制効果と呼ぶ。これには異方性はない。自発磁化の熱揺らぎに起因していると理解される。強磁場によって熱揺らぎが抑えれられ自発磁化の整列度が高まることによって抵抗が小さくなる。この効果はキュリー温度近傍で最大になる。そのため低温下では正常磁気抵抗効果のほうが支配的になり抵抗は大きくなり始めることになる。室温では正常磁気抵抗効果は無視できることがある。
%\\スピン無秩序散乱による磁気抵抗
%\\強磁性軸構造による磁気抵抗
%\\スピン依存トンネル伝導による磁気抵抗
(2)ホール抵抗の外部磁場依存性の図を図3に示した。常磁性体では正常ホール係数、強磁性体では以上ホール係数が主に影響を与えている。
まず常磁性体について考える。幅w厚さd、長さdの試料を考え、x方向に電流$I_x$を流しz方向に磁場$H_z$(磁束密度$B_z$)を印加するとホール電圧$V_H$が生じる。
\begin{align}
V_H=\frac{R_H}{d}I_xB_z
\end{align}•
$R_H$はホール係数である。
さらにホール抵抗は
\begin{align}
\rho=\frac{d\cdot V_H}{I_x}=R_HB_z
\end{align}•
で表される。ホール電圧を求めればホール係数が求まる。キャリアの運動方程式は
\begin{align}
m\frac{d\bm{v}}{dt}=e\bm{E}+e(\bm{v}\times\bm{B})-\frac{m\bm{v}}{\tau}
\end{align}
であり定常電流であること、電流はx方向にのみ流れていることを用いると
\begin{align}
\frac{m}{\tau}v_x=eE_x\\
ev_xB=eE_y
\end{align}
これよりホール電圧は
\begin{align}
V_H=wE_y=wv_xB=w\frac{j_x}{ne}B
\end{align}•
さらにホール係数は
\begin{align}
R_H=\frac{E_y}{j_xB}=\frac{1}{en}
\end{align}•
と求まる。(nはキャリア密度)したがってホール抵抗係数はキャリア密度の逆数に比例することがわかる。これは通常の金属試料でみられ、正常ホール効果と呼ばれる。抵抗は外部磁場に比例している。\\
\quad 一方強磁性体ではホール抵抗(横磁気抵抗)に磁化に比例した項が現れる。
\begin{align}
\rho_H=R_0B+4\pi R_sM
\end{align}•
Mは自発磁化、$R_0$は普通のホール係数、$R_s$は異常ホール係数と呼ばれるものである。このため、抵抗は磁場に比例するだけでなく、磁束密度によって誘起された磁化に依存したふるまいを示す。
\begin{figure}[htbp]
\begin{center}
 \includegraphics [width=80mm]{hole.pdf}
 \end{center}
 \caption{ホール抵抗の磁場依存性}
 \label{fig:one}
\end{figure}
\subsection*{参考文献}
・太田恵三(1973)『磁気工学の基礎II』共立全書\\
\quad  ・近桂一郎・安岡弘志(2001)『実験物理学講座6 磁気測定I』丸善\\
\quad・$\mathrm{http://web.tuat.ac.jp/~katsuaki/hosei/Jiseinyumon.pdf}$\\
\quad・$\mathrm{http://www.ne.phen.mie-u.ac.jp/misc/LLG8.pdf}$\\
\quad・$\mathrm{http://home.sato-gallery.com/research/JEMEA_WG_text20160122.pdf}$\\
\quad・$\mathrm{http://www.murata.com/ja-jp/products/sensor/magnetic/basic/effect}$\\
\quad・$\mathrm{http://home.hiroshima-u.ac.jp/tmatsu/Matsumura/Research_files/trnsprt.pdf}$\\
\quad・$\mathrm{http://supercon.nims.go.jp/matprop/Hall-effect.pdff}$\\
\quad・$\mathrm{https://www.jstage.jst.go.jp/article/materia/48/2/48_55/_pdf}$\\

\end{document}