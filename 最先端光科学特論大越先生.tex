\documentclass{jsarticle}
\usepackage{mathrsfs}
\usepackage{comment}
\usepackage{amsmath}
%\makeatletter
   % \renewcommand{\theequation}{
    %\thesection.\arabic{equation}}
    %\@addtoreset{equation}{section}
\makeatother
\usepackage[dvipdfmx]{graphicx}
\usepackage{bm}


%プリアンブル
\title{最先端光科学特論IV\\ 大越 先生}
\author{理学系研究科物理学専攻\\35176043 小松原望\\nozomu.km@issp.u-tokyo.ac.jp}
\date{2017/8/11}

\begin{document}
\maketitle % タイトルを出力
かったということがわかる。このような解析を遅延ステージを動かしながら行い時間波形をサンプリングしていく。この時間波形をフーリエ変換し、周波数スペクトルに直す。
%%\begin{figure}[t]
%\begin{center}
%\includegraphics [width=80mm]{THz.pdf}
%\end{center}
%\caption{THz波発生素子の表面}
%\label{fig:one}
%\end{figure}
\quad テラヘルツ波の発生及び検出に半導体素子を用いることがある。図1において白い部分に電流を流すことにより紙面垂直方向にTHz波を発生させることができる。青い部分が基盤である。基盤は半絶縁体性ガリウムヒ素の上に合金製の平行伝送線をつけたものとなっている。電極中央の部分が電気双極子の役割を担っている。この間に10Vの電圧を印可し半導体のバンドギャップよりも高いエネルギーを持った光を集光すると半導体の中に光励起されたキャリアが作られる。この時に真ん中のギャップの間でキャリアが加速されてテラヘルツ光が発生する。この時の光は図1において横方向に電場ベクトルを持つ。真ん中のギャップ間はテラヘルツ光の波長に比べて短く、瞬間的に作られたキャリアは集団的に降る会うのでコヒーレントな光となる。\\
参考文献 深澤亮一 「テラヘルツ時間領域分光法と分析化学」2005
\end{document}
