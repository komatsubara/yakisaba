

\documentclass{jsarticle}
\usepackage{mathrsfs}
\usepackage{comment}
\usepackage{amsmath}
%\makeatletter
   % \renewcommand{\theequation}{
    %\thesection.\arabic{equation}}
    %\@addtoreset{equation}{section}
\makeatother
\usepackage[dvipdfmx]{graphicx}
\usepackage{bm}


%プリアンブル
\title{物質科学概論II}
\author{35176043 小松原望 }
%\date{2017/07/25}

\begin{document}
%    \maketitle % タイトルを出力
%\begin{figure}[htbp]
 %\begin{center}
  %\includegraphics [width=80mm]{test2.pdf}
 %\end{center}
 %\caption{FeCo状態密度図}
 %\label{fig:one}
%\end{figure}

ナノ量子情報エレクトロニクス特論Ⅱ\\
\quad35176043 \quad 小松原望\\
(1)まず電気伝導率$\sigma$について考える。$\sigma$はキャリア濃度nに比例し
\begin{eqnarray}
\sigma=nq\mu
\end{eqnarray}
と書ける。$q=\pm e$、$\mu$はキャリアの移動度(易動度)である。\\
\quad つぎに熱伝導率$\kappa$について考える。熱を伝えるものとしてはキャリアと格子振動が存在する。このことを踏まえると熱伝導率は
\begin{eqnarray}
\kappa=\kappa_{ph}+\kappa_{e}
\end{eqnarray}
と書ける。キャリア熱伝導率$\kappa_{e}$と電気伝導率は比例関係にあり(Wiedemann-Franz則)
\begin{eqnarray}
\kappa_{e}=LT\sigma
\end{eqnarray}
という関係がある。ここでLはローレンツ数とよばれる物質によらない定数$(2.5*10^{-8} V^2K^{-2})$、Tは絶対温度である。また、$\kappa_{ph}$はキャリア濃度には依存していない。\\
\quad 電気伝導率$\sigma$、ゼーベック係数の
波数$\bm{k}$、j番目のバンドにある伝導キャリアの数を$f_j(\bm{k},T)$とすると。熱平衡状態ではFermi-Diac1分布に従うとして
\begin{eqnarray}
f^{(0)}_j=\frac{N_j}{\exp\lbrace{\beta(E_j+\epsilon_j{(\bm k})-\mu)+1}\rbrace}\\
\epsilon_j(\bm k)=\sum_{\alpha=1}^3\frac{\hbar^2 k_\alpha^2}{2m_{j\alpha}}
\end{eqnarray}
と表される。($\beta=1/k_B T$、$\mu$は化学ポテンシャル、$E_j$は伝導帯の底のエネルギー、$\epsilon_j(\bm k)$は各方向ごとの有効質量を用いた運動エネルギーである。)
\subsection*{参考文献}
・太田恵三(1973)『磁気工学の基礎II』共立全書\\
\quad  ・近桂一郎・安岡弘志(2001)『実験物理学講座6 磁気測定I』丸善\\
\quad・$\mathrm{http://web.tuat.ac.jp/~katsuaki/hosei/Jiseinyumon.pdf}$\\
\quad・$\mathrm{http://www.ne.phen.mie-u.ac.jp/misc/LLG8.pdf}$\\
\quad・$\mathrm{http://home.sato-gallery.com/research/JEMEA_WG_text20160122.pdf}$\\
\quad・$\mathrm{http://www.murata.com/ja-jp/products/sensor/magnetic/basic/effect}$\\
\quad・$\mathrm{http://home.hiroshima-u.ac.jp/tmatsu/Matsumura/Research_files/trnsprt.pdf}$\\
\quad・$\mathrm{http://supercon.nims.go.jp/matprop/Hall-effect.pdff}$\\
\quad・$\mathrm{https://www.jstage.jst.go.jp/article/materia/48/2/48_55/_pdf}$\\

\end{document}