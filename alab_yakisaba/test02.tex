\documentclass[11pt]{jreport}
\usepackage{amsmath,amssymb,amsfonts}
\usepackage{amsmath}
\usepackage{color}    
%\usepackage{hoge}
\usepackage[dvipdfmx]{graphicx}
\usepackage{jumoline} 
% これより先に color と jumoline を読み込んでおく
%\usepackage{proofread}
\usepackage{comment}
%\noproofreadmark  % 最終原稿のときに有効化する
\newif\ifnoproofreadmark \noproofreadmarkfalse
\def\noproofreadmark{\noproofreadmarktrue}

\def\addspan#1{%
 \ifnoproofreadmark%
   #1%
 \else%
  \textcolor{red}{#1}%
 \fi%
}

\def\delspan#1{%
 \ifnoproofreadmark%
  \empty%
 \else%
  {\Midline{#1}}%
 \fi%
}

\def\delspanx#1{%
 \ifnoproofreadmark%
  \empty%
 \else%
  #1%
 \fi%
}

\def\strikemath#1{%
 \ifnoproofreadmark%
   \empty%
 \else%
   {\setjumocolor{red}\Midline{{\mbox{$\displaystyle #1$}}}}%
 \fi%
}
% \noproofreadmark  % 最終原稿のときに有効化する

\MidlineHeight=0.5ex
\UMOlineThickness=0.2ex % お好みで決める

\newif\ifdraft
\ifdraft
%草案
\else
%本番
\fi
\begin{document}
%a
%\textcolor{blue}{青}
%\Midline{青いうえ}
\addspan{xつ描きくけおk}
\delspan{削除する部分は delspan で,}
\addspan{追加部分は addspan で指定する.}
%
\delspan{数式などのコマンドは {$y = ax + b$} のように中括弧でくくる.}
\addspan{これは delspan も addspan も同様.}

\delspan{%
 delspan の場合,中括弧を開いた直後で改行するのであれば,コメントマーク \% が必要になるっぽい.
}

ディスプレイスタイルの数式は,addspan はできるが delspan できない.打消線を入れるには,数式の \& で区切られた各辺を 
strikemath でくくる.そのままだと,noproofreadmark が指定されたときに数式全体が消えてくれないので,全体を改めて delspanx (delspan ではない)で囲む.(noproofreadmark を使わない場合は必要ない)
  
\delspanx{%
\begin{align}
 \dot{x} & = ax + bu\\
 y & = cx
\end{align}
}%
\addspan{
\begin{align}
 x_{k+1} &= ax_k + bu_k\\
 y_k & = cx_k
\end{align}
}
%\end{comment}
\end{document}