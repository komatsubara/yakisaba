\documentclass[11pt]{jreport}
\usepackage{color}    
%\usepackage{hoge}
%\usepackage{jumoline} 
%\usepackage{proofread} % これより先に color と jumoline を読み込んでおく
% \noproofreadmark  % 最終原稿のときに有効化する
\MidlineHeight=0.5ex
\UMOlineThickness=0.2ex % お好みで決める

\begin{document}
\delspan{削除する部分は delspan で,}\addspan{追加部分は addspan で指定する.}

\delspan{数式などのコマンドは {$y = ax + b$} のように中括弧でくくる.}
\addspan{これは delspan も addspan も同様.}

\delspan{%
 delspan の場合,中括弧を開いた直後で改行するのであれば,コメントマーク \% が必要になるっぽい.
}

ディスプレイスタイルの数式は,addspan はできるが delspan できない.打消線を入れるには,数式の \& で区切られた各辺を 
strikemath でくくる.そのままだと,noproofreadmark が指定されたときに数式全体が消えてくれないので,全体を改めて delspanx (delspan ではない)で囲む.(noproofreadmark を使わない場合は必要ない)
  
\delspanx{%
\begin{align}
 \strikemath{\dot{x}} & \strikemath{= ax + bu}\\
 \strikemath{y} & \strikemath{= cx}
\end{align}
}%
\addspan{
\begin{align}
 x_{k+1} &= ax_k + bu_k\\
 y_k & = cx_k
\end{align}
}